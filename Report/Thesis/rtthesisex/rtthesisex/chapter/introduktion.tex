\chapter{Introduction}\label{cha:intro}
This chapter gives an introduction to the thesis. It describes the background, purpose and goal of the thesis, and also a list of abbreviations and the structure of this report.
\section{Background}
The computationally demanding problems have during a long period of time been solved faster by technical improvements in hardware. However, some limitations have been reached the last decades. Operating frequency of the CPU is no longer significantly improved. Problems relying on single thread performance are limited by three primary technical factors:
The ILP (Instruction-Level Parallelism) wall; CPU instructions in parallel is difficult to exploit.
The memory wall; the increasing gap between CPU speed and memory speed.
The power wall, power and heating problem. The power consumed is increased exponentially with each factorial increase of operating frequency.
Improvements can be found in exploiting parallelism. Either reconstruct the problem or the problem itself is already inherently parallelizable. This trend manifests in development towards use and construction of multi-core microprocessors. The graphical processing unit (GPU) is one such device, originally exploited the inherent parallelism within visual rendering but now is available as a tool for massively parallelizable problems.
Problem statement
Programmers might experience a threshold and slow learning curve to move from sequential programming to thread-parallel programming that is GPU programming. Obstacles involve learning about the hardware architecture and restructure the algorithm or solution. Knowing the limitations and benefits might even provide evidence of not utilizing the GPU and instead choose to work with a multi-core CPU.
Depending on one's preferences, needs and future goals; selecting one platform over the other might be derived from portability needs, hardware requirements, programmability, how well it integrates with other platforms or how well it's supported by the provider or the developer community. Within the range of this thesis, the covered platforms or frameworks are CUDA (Compute Unified Device Architecture), OpenCL (Open Computing Language), DirectCompute and OpenGL Compute Shaders.
Purpose and goal of the thesis work
One goal is to evaluate, select and implement an algorithm suitable for GPGPU (General-purpose computing on graphics processing units).
Implement the same algorithm in important frameworks for GPGPU:
CUDA
OpenCL
DirectCompute (DirectX Compute Shaders)
OpenGL Compute Shaders
The purpose is to compare the different APIs/frameworks by means of benchmarking performance and make qualitative assessments.
