The computational capacity of graphics cards for general-purpose computing have progressed fast over the last decade. A major reason is computational heavy computer games, where standard of performance and high quality graphics constantly rise. Another reason are better suitable technologies for programming the graphics cards. Combined, the product is high raw performance devices and means to access that performance. This thesis investigates some of the current technologies for general-purpose computing on graphics processing units. Technologies are primarily compared by means of benchmarking performance and secondarily by factors concerning programming and implementation. The choice of technology can have a large impact on performance. The benchmark application found the difference of the fastest technology, CUDA, compared to the slowest, OpenCL, to be twice as long execution time. The benchmark application also found out that the older technologies, OpenGL and DirectX, are competitive with CUDA and OpenCL in terms of resulting raw performance.%