\chapter{Evaluation}

\section{Results}

\newcommand{\plotwidth}{{\textwidth} / 2 + 93pt}

The results will be shown for the two graphics cards {\NVCARD} and \AMDCARD, where the technologies were applicable. The tested technologies are shown in table \ref{tab:platform-technologies}. The basics of each technology or library is explained in chapter \ref{cha:technologies}.

\begin{table}
	\centering
	\begin{tabular}{|l|l|}
		\hline
		Platform & Tested technology \\ \hline
		\multirow{3}{*}{\INTELCPU} & \CPP \\
		{} & \OMP \\
		{} & \textit{\FFTW}\tablefootnote{Free software, available at \cite{fftw2015}.} \\ \hline	
		\multirow{5}{*}{\NVCARD} & \CU \\
		{} & \OCL \\
		{} & \DX \\
		{} & \GL \\
		{} & \textit{\CUFFT}\tablefootnote{Available through the \emph{CUDAToolkit} at  \cite{nvidacufft}.} \\ \hline
		\multirow{4}{*}{\AMDCARD} & \OCL \\
		{} & \DX \\
		{} & \GL \\
		{} & \textit{\CLFFT}\tablefootnote{OpenCL FFT library available at \cite{githubclfft}.} \\ \hline
	\end{tabular}
	\caption{Technologies included in the experimental setup.}
	\label{tab:platform-technologies}
\end{table}

The performance measure is total execution time for a single forward transform using two buffers: one input and one output buffer. The implementation input size range is limited by the hardware (graphics card primary memory). However there are some unsolved issues near the upper limit on some technologies on the {\AMDCARD}.

{\CU} is the primary technology and the {\NVCARD} graphics card is the primary platform. All implementations are ported from {\CU}. To compare the implementation, external libraries are included and can be found in italics in the table \ref{tab:platform-technologies}. Note that the {\CLFFT} library failed to be measured in the same manner as the other GPU implementations: the times are measured at host, and short sequences suffer from large overhead.

The experiments are tuned on two parameters, the number of \glspl{thread} per \gls{block} and how large the tile dimensions are in the transpose \gls{kernel}, see chapter \ref{cha:implementation} and table \ref{tab:threads-per-block}.

\subsection{Forward FFT}

The results for a single transform over a $2^{n}$-point sequence are shown in figure \ref{fig:gpu:overview} for the {\NVCARD}, {\AMDCARD} and {\INTELCPU}.

%1D TIME GTX
\begin{figure}
	\centering
	\subfloat[\NVCARD]{	
		\includestandalone[width=\plotwidth]{plots/gtx-overview}
	}
	\vfill
	\subfloat[\AMDCARD]{
		\includestandalone[width=\plotwidth]{plots/r260x-overview}
	}
	\caption{Overview of the results of a single forward transform. The {\CLFFT} was timed by host synchronization resulting in an overhead in the range of $60{\micro}s$. Lower is faster.}
	\label{fig:gpu:overview}
\end{figure}

The {\CU} implementation was the fastest on the {\NVCARD} over most sequences. The {\OCL} implementation was the only technology that could run the whole test range on the {\AMDCARD}. {\DX} was limited to $2^{24}$ points and OpenGL to $2^{23}$ points. A normalized comparison using {\CU} and {\OCL} is shown in figure \ref{fig:gpu:implementation}.

\begin{figure}
	\centering
	\subfloat[\NVCARD\label{fig:gpu:implementation:gtx}]{	
		\includestandalone[width=\plotwidth]{plots/gtx-implementation}
	}
	\vfill
	\subfloat[\AMDCARD\label{fig:gpu:implementation:r260x}]{
		\includestandalone[width=\plotwidth]{plots/r260x-implementation}
	}	
	\caption{Performance relative {\CU} implementation in \ref{fig:gpu:implementation:gtx} and {\OCL} in \ref{fig:gpu:implementation:r260x}. Lower is better.}
	\label{fig:gpu:implementation}
\end{figure}

The experimental setup for the CPU involved low overhead and the short sequences could not be measured accurately, shown as $0{\micro}s$ in the figures. Results from comparing the sequential {\CPP} and multi-core {\OMP} implementation with {\CU} are shown in figure \ref{fig:gtx:cpu}. {\FFTW} was included and demonstrated how an optimized (per $n$-point length) sequential CPU implementation perform.

\begin{figure}
	\centering
	\includestandalone[width=\plotwidth]{plots/gtx-cpu}
	\caption{Performance relative {\CU} implementation on {\NVCARD} and {\INTELCPU}.}
	\label{fig:gtx:cpu}
\end{figure}

Results from comparing the implementations on the different graphics cards are shown in figure \ref{fig:gpu-comparison}. The results are normalized on the result of the tests on the {\NVCARD}.

\begin{figure}
	\centering
	\includestandalone[width=\plotwidth]{plots/gpu-comparison}
	\caption{Comparison between {\AMDCARD} and {\NVCARD}.}
	\label{fig:gpu-comparison}
\end{figure}

{\DX}, {\GL}, and {\OCL} was supported on both graphics cards, the results of normalizing the resulting times with the time of the {\OCL} implementation is shown in figure \ref{fig:gpu-comparison-tech}.

\begin{figure}
	\centering
	\includestandalone[width=\plotwidth]{plots/gpu-comparison-tech}
	\caption{Performance relative {\OCL} accumulated from both cards.}
	\label{fig:gpu-comparison-tech}
\end{figure}

% 2D Figures

\newpage

\subsection{FFT 2D}

The equivalent test was done for \gls{2D}-data represented by an image of $m{\times}m$ size. The image contained three channels (red, green, and blue) and the transformation was performed over one channel. Figure \ref{fig:gpu:overview-2d} shows an overview of the results of images sizes ranging from $2^{6}{\times}2^{6}$ to $2^{13}{\times}2^{13}$.

\begin{figure}
	\centering
	\subfloat[\NVCARD]{	
		\includestandalone[width=\plotwidth]{plots/gtx-overview-2d}
	}
	\vfill
	\subfloat[\AMDCARD]{
		\includestandalone[width=\plotwidth]{plots/r260x-overview-2d}
	}
	\caption{Overview of the results of measuring the time of a single 2D forward transform.}
	\label{fig:gpu:overview-2d}
\end{figure}

All implementations compared to {\CU} and {\OCL} on the {\NVCARD} and {\AMDCARD}respectively are shown in \ref{fig:gpu:implementation-2d}. The {\GL} implementation failed at images larger then $2^{11}{\times}2^{11}$ points.

\begin{figure}
	\centering
	\subfloat[\NVCARD\label{fig:gpu:implementation-2d:gtx}]{	
		\includestandalone[width=\plotwidth]{plots/gtx-implementation-2d}
	}
	\vfill
	\subfloat[\AMDCARD\label{fig:gpu:implementation-2d:r260x}]{
		\includestandalone[width=\plotwidth]{plots/r260x-implementation-2d}
	}	
	\caption{Time of 2D transform relative {\CU} in \ref{fig:gpu:implementation-2d:gtx} and {\OCL} in \ref{fig:gpu:implementation-2d:r260x}.}
	\label{fig:gpu:implementation-2d}
\end{figure}

The results of comparing the \gls{GPU} and \gls{CPU} handling of a \gls{2D} forward transform is shown in figure \ref{fig:gtx:cpu-2d}.

\begin{figure}
	\centering
	\includestandalone[width=\plotwidth]{plots/gtx-cpu-2d}
	\caption{Performance relative {\CU} implementation on {\NVCARD} and {\INTELCPU}.}
	\label{fig:gtx:cpu-2d}
\end{figure}

Comparison of the two cards are shown in figure \ref{fig:gpu-comparison-2d}.

\begin{figure}
	\centering
	\includestandalone[width=\plotwidth]{plots/gpu-comparison-2d}
	\caption{Comparison between {\AMDCARD} and {\NVCARD} running 2D transform.}
	\label{fig:gpu-comparison-2d}
\end{figure}

{\DX}, {\GL} and {\OCL} was supported on both graphics cards, the results of normalizing the resulting times with the time of the {\OCL} implementation is shown in figure \ref{fig:gpu-comparison-tech-2d}.

\begin{figure}
	\centering
	\includestandalone[width=\plotwidth]{plots/gpu-comparison-tech-2d}
	\caption{Performance relative {\OCL} accumulated from both cards.}
	\label{fig:gpu-comparison-tech-2d}
\end{figure}

\newpage

\section{Discussion}

The foremost known technologies for GPGPU, based on other resarch-interests, are {\CU} and {\OCL}. The comparisons from earlier work have focused primarily on the two \cite{fang2011comprehensive, park2011design, su2012overview}. Bringing {\DX} (or Direct3D Compute Shader) and {\GL} Compute Shader to the table makes for an interesting mix since the result from the experiment is that both are strong alternatives in terms of raw performance.

The most accurate and fair comparison with a GPU is when number of data is scaled up, the least amount of elements should be in the order of $2^{12}$. By not fully saturating the GPUs streaming multiprocessors, there is no gain from moving from the CPU. One idea is to make sure that even if the sequences are short, they should be calculated in batches. The results from running the benchmark application on small sets of data are more or less discarded in the evaluation.

\subsubsection{The CPU vs GPU}

The implementation aimed at sequences of two-dimensional data was however successful at proving the strength of the GPU versus the CPU. The difference in execution time of the CPU and GPU is a factor of 40 times slower when running a 2D FFT over large data. Compared to the multi-core {\OMP} solution, the difference is still a factor of 15 times slower. Even the optimized {\FFTW} solution is a factor of 10 times slower. As a side note, the {\CUFFT} is 36 times faster than {\FFTW} on large enough sequences, they do use the same strategy (build an execution plan based on current hardware and data size) and likely use different hard-coded unrolled FFTs for smaller sizes.

\subsubsection{The GPU}

The unsurprising result from the experiments is that {\CU} is the fastest technology on {\NVCARD}, but only with small a margin. What might come as surprise, is the strength of the {\DX} implementation. Going head-to-head with {\CU} (only slightly slower) on the {\NVCARD}, and performing equally (or slightly faster) than {\OCL}.

{\GL} is performing on par with {\DX} on the {\AMDCARD}. The exception is that the long sequences fails with {\GL}-solution on the {\AMDCARD}, sequences otherwise working on the {\NVCARD}. The performance of the {\GL} tests are equal or better then {\OCL} in 1D, but outperforming {\OCL} in 2D.

The biggest surprise is actually the {\OCL} implementation. Falling behind by a relatively big margin on both graphics cards. This large margin was not anticipated based on other papers in comparisons. Effort has been made to assure that the code does in fact run fairly compared to the other technologies. The ratio for {\OCL} versus {\CU} on long sequences are about $1.6$ and $1.8$ times slower for 1D and 2D respectively on the {\NVCARD}. The figure \ref{fig:gpu-comparison-tech} and \ref{fig:gpu-comparison-tech-2d} shows that {\DX} is a factor of about $0.8$ of the execution-time of {\OCL}. {\GL} beats the {\OCL} solution and is on par with {\DX} in terms of execution time. However, fails to execute the longest sequences on the {\AMDCARD}. The one thing that goes in favor of {\OCL} is that the implementation did scale without problem: All sequences were computed as expected. The figures \ref{fig:gpu:implementation:r260x} and \ref{fig:gpu:implementation-2d:r260x} shows that something happened with the other implementations, even {\CLFFT} had problem with the last sequence. {\GL} and {\DX} could not execute all sequences.

\subsubsection{External libraries}

Both {\FFTW} on the CPU and {\CUFFT} on the {\NVCARD} proved to be very mature and optimized solutions, far faster then any of my implementations on respective platform. Not included in the benchmark implementation is a {\CPP} implementation that partially used the concept of the {\FFTW} (a decomposition with hard-coded unrolled FFTs for short sequences) and was fairly fast at short sequences compared to {\FFTW}. Scalability proved to be poor and provided very little gain in time compared to much simpler implementations such as the constant geometry algorithm.

{\CUFFT} proved stable and much faster than any other implementation on the GPU. The GPU proved stronger than the CPU at data sizes of $2^{12}$ points or larger, this does not include memory transfer times. Comparing {\CUFFT} with {\CLFFT} was possible on the {\NVCARD}, but that proved only that {\CLFFT} was not at all written for that architecture and was much slower at all data sizes. A big problem when including the {\CLFFT} library was that measuring by events on the device failed, and measuring at the host included an overhead. Short to medium-long sequences suffered much from the overhead, a quick inspection suggests of close to $60{\micro}s$ (comparing to total runtime of the {\OCL} at around $12{\micro}s$ for short sequences). Not until sequences reached $2^{16}$ elements or greater could the library beat the implementations in the application. The results are not that good either, a possible explanation is that the {\CLFFT} is not at all efficient at executing transforms of small batches, the blog post at \cite{amd2015performance} suggest a completely different result when running in batch and on GPUs designed for computations. The conclusions in this work are based on cards targeting the gaming consumer market and variable length sequences.

\subsection{Qualitative assessment}

When working with programming, raw performance is seldom the only requirement. This subsection will provide qualitative based assessments of the technologies used.

\subsubsection{Scalability of problems}

The different technologies are restricted in different ways. {\CU} and {\OCL} are device limited and suggest polling the device for capabilities and limitations. {\DX} and {\GL} are standardized with each version supported. An example of this is the shared memory limit: {\CU} allowed for full access, whereas {\DX} was limited to a \gls{API}-specific size and not bound by the specific device. The advantage of this is the ease of programming with {\DX} and {\GL} when knowing that a minimum support is expected at certain feature support versions.

Both {\DX} and {\GL} had trouble when data sizes grew, no such indications when using {\CU} and {\OCL}.

\subsubsection{Portability}

{\OCL} have a key feature of being portable and open for many architecture enabling computations. However, as stated in \cite{fang2011comprehensive, du2012cuda}, performance is not portable over platforms but can be addressed with auto-tuning at the targeted platform. There were no problems running the code on different graphic cards on either {\OCL} or {\DX}. {\GL} proved to be more problematic with two cards connected to the same host. The platform-specific solution using either \gls{OS} tweaking or specific device {\GL} expansions made {\GL} less convenient as a GPGPU platform. {\CU} is a proprietary technology and only usable with NVIDIAs own hardware.

Moving from the \gls{GPU}, the only technology is {\OCL} and here is where it excels among the others. This was not in the scope of the thesis however it is worth noting that it would be applicable with minor changes in the application.

\subsubsection{Programmability}

The experience of this work was that {\CU} was by far the least complicated to implement. The fewest lines of code needed to get started and compared to {\CPP} there were fewer limitations. The {\CU} community and online documentation is full of useful information, finding solutions to problems was relatively easy. The documentation\footnote{\textit{https://docs.nvidia.com/cuda/cuda-c-programming-guide/}} provided guidance for most applications.

{\OCL} implementation was not as straight forward as {\CU}. The biggest difference is the setup. Some differences in the setup are:
\begin{itemize}
	\item Device selection is not needed actively in {\CU}
	\item Command queue or stream is created by default in {\CU}
	\item {\CU} creates and builds the \gls{kernel} run-time instead of compile-time.
\end{itemize}
Both {\DX} and {\GL} follows this pattern, although they inherently suffer from graphic specific abstractions. The experience was that creating and handle memory-buffers was more prone to mistakes. Extra steps was introduced to create and use the memory in a compute shader compared to a {\CU} and {\OCL}-kernel.

The biggest issue with {\GL} is the way the device is selected, it is handled by the \gls{OS}. Firstly, in the case of running \emph{Windows 10}, the card had to be connected to a screen. Secondly, that screen needed to be selected as the primary screen. This issue is also a problem when using services based on \gls{RDP}. \gls{RDP} enables the user to log in to a computer remotely. This works for the other technologies but not for {\GL}. Not all techniques for remote access have this issue, it is convenient if the native tool in the \gls{OS} support \gls{GPGPU} features such as selecting the device, especially when running a benchmarking application.

\subsection{Method}

The first issue that have to be highlighted is the fact that \gls{1D} \gls{FFT}s were measured by single sequence execution instead of executing sequences in batch. The \gls{2D} \gls{FFT} implementation did inherently run several \gls{1D} sequences in batch and provided relevant results. One solution would have been to modify the \gls{2D} transformation to accept a batch of sequences organized as \gls{2D} data. The second part of performing column-wise transformations would then be skipped.

The two graphics cards used are not each others counterparts, the releases differs 17 months ({\NVCARD} released May 10, 2012 compared to {\AMDCARD} in October 8, 2013). The recommended release price hints the targeted audience and capacity of the cards, the {\NVCARD} was priced at \$400 compared to the {\AMDCARD} at \$139. Looking at the {\OCL} performance on both cards as seen in figure \ref{fig:gpu-comparison-2d}, revealed that medium to short sequences are about the same, but longer sequences goes in favour of the {\NVCARD}.

\subsubsection{Algorithm}

The \gls{FFT} implementation was rather straight forward and without heavy optimization. The implementation lacked in performance compared to the NVIDIA developed {\CUFFT} library. Figure \ref{fig:gpu:implementation-2d:gtx} shows {\CUFFT} to perform three to four times faster then the benchmark application. Obviously there is a lot to improve. This is not a big issue when benchmarking, but it removes some of the credibility of the algorithm as something practically useful.

By examining the code with NVIDIA Nsight\footnote{A tool for debugging and profiling CUDA applications.}, the bottleneck was the memory access pattern when outputting data after bit-reversing the index. There were no coalesced accesses and bad data locality. There are algorithms that solves this by combining the index transpose operations with the FFT computation as in \cite{govindaraju2008high}.

Some optimizations were attempted during the implementation phase and later abandoned. The reason why attempts was abandoned was the lack of time or no measurable improvement (or even worse performance). The use of shared memory provide fewer global memory accesses, the shared memory can be used in optimized ways such as avoiding banking conflicts\footnote{Good access pattern allows for all threads in a warp to read in parallel, one per memory bank at a total of 32 banks on {\NVCARD}, a banking conflict is two threads in a warp attempting to read from the same bank and becomes serialized reads.}. This was successfully tested on the {\CU} technology with no banking conflicts at all, but gave no measurable gain in performance. The relative time gained compared to global memory access was likely to small or was negated by the fact that an overhead was introduced and more shared memory had to be allocated per block.

The use of shared memory in the global \gls{kernel} combined with removing the better part of multiple host-synchronizations is a potentially good optimization. The time distribution during this thesis did not allow further optimizations. The attempts to implement this in short time were never completed successfully. Intuitively guessed, the use of shared memory in the global kernel would decrease global memory accesses and reduce the total number of kernel launches to $\ceil{\log_{2}(\frac{N}{N_{block}})}$ compared to $\log_{2}(N) - \log_{2}(N_{block}) + 1$.

\subsubsection{Wider context}

Since \gls{GPU} acceleration can be put to great use in large computing environments, the fastest execution time and power usage is important. If the same hardware performs faster with another technology the selection or migration is motivated by reduced power costs. Data centers and \gls{HPC} is becoming a major energy consumer globally, and future development must consider all energy saving options.

\section{Conclusions}

\subsection{Benchmark application}

The \gls{FFT} algorithm was successfully implemented as benchmark application in all technologies. The application provided parallelism and enough computational complexity to take advantage of the \gls{GPU}. This is supported by the increase in speed compared to the \gls{CPU}. The benchmark algorithm executed up to a factor of $40$ times faster on the GPU as seen in table \ref{tab:cu-vs-cpu} where {\CU} is compared to the CPU implementations.

\begin{table}
	\centering	
	\begin{tabular}{|l|r|r|}
		\hline
		Implementation & 1D & 2D \\ \hline
		{\CU} & 1 & 1 \\ \hline
		{\OMP} & ${\times}13$ & ${\times}15$ \\ \hline
		{\CPP} & ${\times}18$ & ${\times}40$ \\ \hline
	\end{tabular}
	\caption{Table comparing {\CU} to CPU implementations.}
	\label{tab:cu-vs-cpu}
\end{table}

\subsection{Benchmark performance}

Benchmarking on the {\NVCARD} performing a \gls{2D} forward-transformation showed the following rank:
\begin{enumerate}	
	\item {\CU}
	\item {\DX}
	\item {\GL}
	\item {\OCL}
\end{enumerate}

Benchmarking on the {\AMDCARD} performing a \gls{2D} forward-transformation showed the following rank:
\begin{enumerate}	
	\item {\DX}
	\item {\GL}\footnote{Failed to compute sequences longer then $2^{23}$ elements.}
	\item {\OCL}\footnote{Performance very close to or equal to {\DX} when sequence reached $2^{24}$ elements or more.}
\end{enumerate}
The ranking reflects the results of a combined performance relative {\OCL}, where {\DX} average at a factor of $0.8$ the speed of {\OCL}.

\subsection{Implementation}

The {\CU} implementation had a relative advantage in terms of code size and complexity. The necessary steps to setup a kernel before executing was but a fraction of the other technologies. {\OCL} needs runtime compiling of the kernel and thus require many more steps and produces a lot more setup code. Portability is in favour of {\OCL}, but was not examined further (as in running the application on the \gls{CPU}). The {\DX} setup was similar to {\OCL} with the addition of some more specifications required. {\GL} followed this pattern but did lack the support to select device if several was available.

The kernel code was fairly straight forward and was relatively easy to port from {\CU} to any other technology. Most issues could be traced back to the memory buffer handling and related back to the setup phase or how buffers needed to be declared in-code. {\CU} offered the most in terms of support to a {\CPP} like coding with fewer limitations than the rest.

\section{Future work}

This thesis work leave room for expanding with more test applications and improve the already implemented algorithm.

\subsection{Application}

The \gls{FFT} algorithm is implemented in many practical applications, the performance tests might give different results with other algorithms. The \gls{FFT} is very easy parallelized and put great demand on the memory by making large strides. It would be of interest to expand with other algorithms that puts more strain on the use of arithmetic operations.

\subsubsection{FFT algorithms}

The benchmark application is much slower than the external libraries for the \gls{GPU}, the room for improvements ought to be rather large. One can not alone expect to beat a mature and optimized library such as {\CUFFT}, but one could at least expect a smaller difference in performance in some cases. Improved or further use of shared memory and explore a precomputed twiddle factor table would be a topic to expand upon. Most important would probably be to examine how to improve the access pattern towards the global memory.

For the basic algorithm there are several options to remove some of the overhead when including the bit-reversal as a separate step by selecting an algorithm with different geometry.

Based on {\CUFFT} that uses {\CTALG} and Bluestein's algorithm, a suggested extension would be to expand to other then $2^{k}$ sizes and implement to compare any size of sequence length.

\subsection{Hardware}

\subsubsection{More technologies}

The graphic cards used in this thesis are at least one generation old compared to the latest graphic cards as of late 2015. It would be interesting to see if the cards have the same differences in later generations and to see how much have been improved over the generations. It is likely that the software drivers are differently optimized towards the newer graphic cards.

The DirectX 12 \gls{API} was released in the fourth quarter of 2015 but this thesis only utilized the DirectX 11 API drivers. The release of \emph{Vulkan}, comes with the premise much like DirectX 12 with high performance and more low level interaction. In a similar way AMDs \emph{Mantle} is an alternative to Direct3D with the aim of reducing overhead. Most likely, the (new) hardware will support the newer APIs in a more optimized way during the coming years.

\subsubsection{Graphics cards}

The {\NVCARD} have the \textit{Kepler} micro architecture. The model have been succeeded by booth the 700 and 900 GeForce series and the micro architecture have been followed by \textit{Maxwell} (2014). Both Kepler and Maxwell uses 28nm design. The next micro architecture is \textit{Pascal} and is due in 2016. Pascal will include 3D memory and \gls{HBM2} that will move onto the same package as the GPU and greatly improve memory bandwidth and total size. Pascal will use 16nm transistor design that will grant higher speed and energy efficiency.

The {\AMDCARD} have the \gls{GCN} 1.1 micro architecture and have been succeeded by the Radeon Rx 300 Series and \gls{GCN} 1.2. The latest graphic cards in the Rx 300 series include cards with \gls{HBM} and will likely be succeeded by \gls{HBM2}. The {\AMDCARD} is not target towards the high-end consumer so it would be interesting to see the performance with a high-end AMD GPU.

\subsubsection{\INTELCPU}

The used {\INTELCPU} have four real cores but can utilize up to eight threads in hardware. Currently the trend is to utilize more cores per die when designing new CPUs. The release of Intel Core i7-6950X and i7-6900K targeting the high-end consumer market will have 10 and 8 cores. The i7-6950X is expected some time in the second quarter in 2016.

Powerful multi-core \gls{CPU}s will definitely challenge GPUs in terms of potential raw computing capability. It would make for an interesting comparison by using high-end consumer products of the newest multi-core CPUs and GPUs. This work was made with processing units from the same generation (released in 2012-2013) and the development in parallel programming have progressed and matured since.