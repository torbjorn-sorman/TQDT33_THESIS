\chapter{Introduction}\label{cha:intro}
This chapter gives an introduction to the thesis. It describes the background, purpose and goal of the thesis, and also a list of abbreviations and the structure of this report.

\section{Background}
The computationally demanding problems have during a long period of time been solved faster by technical improvements in hardware. However, some limitations have been reached the last decades. Operating frequency of the CPU is no longer significantly improved. Problems relying on single thread performance are limited by three primary technical factors:
\begin{enumerate}
	\item The Instruction-Level Parallelism (ILP) wall
	\item The memory wall
	\item The power wall
\end{enumerate}

The first wall states that its hard to further exploit simultanious CPU instructions, techniques like instruction pipelining, superscalar execution and VLIW exists but complexity and latency of hardware reduces the benefits. Related to the first is second wall, the gap between CPU speed and memory access time, that may cost several hundreds of CPU cykles if accessing primary memory. The third wall is power and heating problem. The power consumed is increased exponentially with each factorial increase of operating frequency.

Improvements can be found in exploiting parallelism. Either reconstruct the problem or the problem itself is already inherently parallelizable. This trend manifests in development towards use and construction of multi-core microprocessors. The graphical processing unit (GPU) is one such device, originally exploited the inherent parallelism within visual rendering but now is available as a tool for massively parallelizable problems.

\section{Problem statement}
Programmers might experience a threshold and slow learning curve to move from sequential to thread-parallel programming that is GPU programming. Obstacles involve learning about the hardware architecture and restructure the application. Knowing limitations and benefits might even provide reason to not utilize the GPU and instead choose to work with a multi-core CPU.

Depending on one's preferences, needs and future goals; selecting one technology over the other might be derived from portability or hardware requirements, programmability, how well it integrates with other frameworks or APIs or how well it's supported by the provider or developer community. Within the range of this thesis, the covered technologies are CUDA (Compute Unified Device Architecture), OpenCL (Open Computing Language), DirectCompute (API within DirectX) and OpenGL Compute Shaders.

\section{Purpose and goal of the thesis work}
One goal is to evaluate, select and implement an application suitable for GPGPU (General-purpose computing on graphics processing units).

Implement the same application in important technologies for GPGPU:
\begin{itemize}
	\item CUDA
	\item OpenCL
	\item DirectCompute
	\item OpenGL Compute Shaders
\end{itemize}

The purpose is to compare the different technologies by means of benchmarking performance and make relevant qualitative assessments.