\chapter{Technologies}

\newcommand{\procwidth}{{\textwidth * 3 / 4}}

Five different multi-core technologies are used in this study. One is a proprietary parallel computing platform and API, namely {\CU}. \textit{Compute Shaders} in {\GL} and {\DX} are parts of graphic programming languages but have a fairly general structure and allows for general computing. {\OCL} aims at any heterogeneous multi-core system and is used in this study solely on the GPU. To compare with the CPU, {\OMP} is included as an effective way to parallelize sequential {\CPP}-code.

\section{CUDA}

CUDA is an acronym for Compute Unified Device Architecture, developed by NVIDIA and released in 2006. {\CU} is an extension of the {\CPP} language and have its own compiler. {\CU} supports the functionality to execute kernels, modify the graphic card RAM memory and the use of several optimized function libraries such as \textit{cuBLAS} ({\CU} implementation of BLAS, Basic Linear Algebra Subprograms) or \textit{cuFFT} ({\CU} implementation of FFT).

A program running on the GPU is called a kernel. The GPU is referred to as the \textit{device} and the the CPU is called the \textit{host}. To run a {\CU} kernel, all that is needed is to declare the program with the function type specifier \code{\_\_global\_\_}, for others see table \ref{tab:cuda:func-types}, and call it from the host with launch arguments. The kernel execution call includes specifying the thread organization. Threads are organized in \emph{blocks} that in turn are specified within a \emph{grid}. Both the block and grid can be used as 1-, 2- or 3-dimensional to help the addressing in a program. These can be accessed within a kernel by the structures \code{blockDim} and \code{gridDim}. Thread and block identification is done with \code{threadIdx} and \code{blockIdx}.

All limitations can be polled from the device and all devices are have a minimum feature support called \emph{Compute capability}. The compute capability aimed at in this thesis is $3.0$ and includes the GPU models starting with \emph{GK} or later (\emph{Tegra} and \emph{GM}).

CUDA exposes intrinsic functions on the device and a variety of fast math functions, optimized single-precision operations, denoted with the suffix -\emph{f}. In the CUDA example in figure \ref{lst:sample:global:cu} the trigonometric function \code{\_\_sincosf} is used to calculate both $\sin{\alpha}$ and $\cos{\alpha}$ at the same time.

\begin{table}
	%\includestandalone[width=\textwidth]{tables/cuda-function-types}
	\centering
	\begin{tabular}{|l|l|l|}
	\hline
	Function type & Executed on & Callable from \\ \hline
	\texttt{\_\_device\_\_} & Device & Device \\ \hline
	\texttt{\_\_global\_\_} & Device & Host \\ \hline
	\texttt{\_\_host\_\_} & Host & Host \\ \hline
\end{tabular}
	\caption{Table of function types in CUDA.}
	\label{tab:cuda:func-types}
\end{table}

\begin{figure}
	\centering
	\fbox{\includestandalone[width=\procwidth]{code/sample/cu}}
	\caption{Example {\CU} global kernel}
	\label{lst:sample:global:cu}	
\end{figure}

\section{OpenCL}

{\OCL} is a framework and open standard for writing programs that executes on a many multi-core platforms such as CPUs, GPUs and FPGAs among other processors and hardware accelerators. {\OCL} uses a similar structure as {\CU}, the language is based on \emph{C99} when programming a device. The standard is supplied by the \emph{The Khronos Groups} and the implementation is supplied by the manufacturing company or device vendor such as AMD, INTEL or NVIDIA.

{\OCL} views the system from a perspective where computing resources (CPU or other accelerators) are a number of \emph{compute devices} attached to a host (a CPU). The programs executed on a compute device is called a kernel. Programs in the {\OCL} language are intended to be compiled at run-time to preserve portability between implementations from various host devices.

The {\OCL} kernel are compiled and executed on the host and then enqueued at a specific device. The kernel function accessible by the host to enqueue is specified with \code{\_\_kernel}. Data residing in global memory is specified in the parameter list by \code{\_\_global} and local memory have the specifier \code{\_\_local}. The {\CU} threads are in {\OCL} terminology called \emph{Work-items} and they are organized in \emph{Work-groups}.

\begin{figure}
	\centering
	\fbox{\includestandalone[width=\procwidth]{code/sample/ocl}}
	\caption{{\OCL} global kernel}
	\label{lst:sample:global:ocl}	
\end{figure}

Similarly to {\CU} the host application can poll the device for its capabilities and use some fast math function. The equivalent {\CU} kernel in figure \ref{lst:sample:global:cu} is implemented in {\OCL} in figure \ref{lst:sample:global:ocl} and displays small differences. The fast math function \code{\_\_sincosf} is swapped with \code{sincos}.
%, some other terminology differences is found in table \ref{tab:sample:terms-cu-ocl}.

%\begin{table}
%	\centering
%	\begin{tabular}{|l|l|l|}
%		\hline
%		\CU & \OCL & \DX \\ \hline
%		\code{gridDim} & \code{get\_num\_groups()} & - \\ \hline
%		\code{blockDim} & \code{get\_local\_size()} & - \\ \hline
%		\code{blockIdx} & \code{get\_group\_id()} & \code{groupID} \\ \hline
%		\code{threadIdx} & \code{get\_local\_id} & \code{threadIDInGroup} \\ \hline
%		\code{blockDim $\times$ blockIdx + threadIdx} & \code{get\_global\_id()} & \code{groupID.x $\times$ GROUP\_SIZE\_X + threadIDInGroup} \\ \hline
%		\code{gridDim $\times$ blockDim} & \code{get\_global\_size()} & \code{GROUP\_SIZE\_X $\times$ GRID\_DIM\_X} \\ \hline	
%	\end{tabular}
%	\caption{Terminology differences between the technologies, the dash (-) signifies that the struct was not applicable and instead the shader/kernel code was modified before compiled}
%	\label{tab:sample:terms-cu-ocl}
%\end{table}

\section{DirectCompute}

Microsoft {\DX} is an API that supports GPGPU on Microsoft's Windows OS (Vista, 7, 8, 10). {\DX} is part of the \emph{DirectX} collection of APIs. The initial release was with DirectX 11 API and have similarities with both {\CU} and {\OCL}. {\DX} is designed and implemented with \emph{High-Level Shading Language} (HLSL) for its kernel equivalent the \emph{compute shader}. The compute shader is not like the other type of shaders, like the vertex or pixel shaders used in graphic processing.

The major differences from {\CU} and {\OCL} in implementing a compute shader comparing to a kernel are the lack of C-like parameters (a \emph{constant buffer} is used instead, each value is stored in a read-only data structure). The setup share much of the work that has to be done with {\OCL} and is compiled at run-time. The block dimensions is built in as a constant value in the compute shader and the block dimensions are specified at shader dispatch/execution.

As the code example demonstrated in figure \ref{lst:sample:global:dx} the shader body is similar to that of {\CU} and {\OCL}.

\begin{figure}
	\centering
	\fbox{\includestandalone[width=\procwidth]{code/sample/dx}}
	\caption{{\DX} global kernel}
	\label{lst:sample:global:dx}	
\end{figure}

\section{OpenGL}

Open Graphics Library {\GL} share much of the same graphics inheritance as {\DX} but also provides a compute shader that breaks out of the graphics pipeline. Analogous to HLSL, {\GL} programs are implemented with OpenGL Shading Language (GLSL). The minor differences between the two are subtle but include how arguments are passed and the use of specifiers.

\begin{figure}
	\centering
	\fbox{\includestandalone[width=\procwidth]{code/sample/gl}}
	\caption{{\GL} global kernel}
	\label{lst:sample:global:gl}	
\end{figure}

\section{OpenMP}%
%
\begin{figure}%
	\centering%
	\fbox{\includestandalone[width=\procwidth]{code/sample/omp}}%
	\caption{{\OMP} procedure completing one stage}%
	\label{lst:sample:global:omp}%
\end{figure}%

\section{Related work}

Write about other benchmark papers.

Read more about it on NVIDIA CUDA resources.