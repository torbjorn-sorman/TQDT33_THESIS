\chapter{Technologies}

\newcommand{\procwidth}{{\textwidth * 3 / 4}}

Five different multi-core technologies are used in this study. One is a proprietary parallel computing platform and API, namely {\CU}. \textit{Compute Shaders} in {\GL} and {\DX} are parts of graphic programming languages but have a fairly general structure and allows for general computing. {\OCL} aims at any heterogeneous multi-core system and is used in this study solely on the GPU. To compare with the CPU, {\OMP} is included as a fast and easy way to parallelize sequential {\CPP} -code.

\section{CUDA}

CUDA is an acronym for Compute Unified Device Architecture, developed by NVIDIA and released in 2006. {\CU} is an extension of the {\CPP} language and have its own compiler. {\CU} supports the functionality to execute kernels, modify the graphic card RAM memory and the use of several optimized function libraries such as \textit{cuBLAS} ({\CU} implementation of BLAS, Basic Linear Algebra Subprograms) or \textit{cuFFT} ({\CU} implementation of FFT).

A program running on the GPU is called a kernel. The GPU is referred to as the \textit{device} and the the CPU is called the \textit{host}. To run a {\CU} kernel all that is needed is to declare the program with a function type specifier, see table \ref{tab:cuda:func-types}, and call it from the host with launch parameters.

\begin{table}
	%\includestandalone[width=\textwidth]{tables/cuda-function-types}
	\centering
	\begin{tabular}{|l|l|l|}
	\hline
	Function type & Executed on & Callable from \\ \hline
	\texttt{\_\_device\_\_} & Device & Device \\ \hline
	\texttt{\_\_global\_\_} & Device & Host \\ \hline
	\texttt{\_\_host\_\_} & Host & Host \\ \hline
\end{tabular}
	\caption{Table of function types in CUDA.}
	\label{tab:cuda:func-types}
\end{table}

\begin{figure}
	\centering
	\fbox{\includestandalone[width=\procwidth]{code/sample/cu}}
	\caption{Example {\CU} global kernel}
	\label{lst:sample:global:cu}	
\end{figure}

\section{OpenCL}

\begin{figure}
	\centering
	\fbox{\includestandalone[width=\procwidth]{code/sample/ocl}}
	\caption{{\OCL} global kernel}
	\label{lst:sample:global:ocl}	
\end{figure}

\section{DirectCompute}

\begin{figure}
	\centering
	\fbox{\includestandalone[width=\procwidth]{code/sample/dx}}
	\caption{{\DX} global kernel}
	\label{lst:sample:global:dx}	
\end{figure}

\section{OpenGL Compute Shader}

\begin{figure}
	\centering
	\fbox{\includestandalone[width=\procwidth]{code/sample/gl}}
	\caption{{\GL} global kernel}
	\label{lst:sample:global:gl}	
\end{figure}

\section{OpenMP}%
%
\begin{figure}%
	\centering%
	\fbox{\includestandalone[width=\procwidth]{code/sample/omp}}%
	\caption{{\OMP} procedure completing one stage}%
	\label{lst:sample:global:omp}%
\end{figure}%

\section{Related work}

Write about other benchmark papers.

Read more about it on NVIDIA CUDA resources.