\chapter{Theory}

This chapter will give an introduction to the \gls{FFT} algorithm and a brief introduction of the \gls{GPU}.

\section{Graphics Processing Unit}

A GPU is traditionally specialized hardware for efficient manipulation of computer graphics and image processing\cite{owens2008gpu}. The inherent parallel structure of images and graphics makes them very efficient at some more general problems where parallelism can be exploited. The concept of \gls{GPGPU} is solving a problem on the \gls{GPU} platform instead of a multi-core \gls{CPU} system.

\subsection{GPGPU}

In the early days of \gls{GPGPU} one had to know a lot about computer graphics to compute general data. The available APIs was created for graphics processing. The dominant \gls{API}s was OpenGL and DirectX. \gls{HLSL} and \gls{GLSL} made the step easier but it still generated code into the \gls{API}s.

A big change was when NVIDIA released {\CU} and together with new hardware made it possible to use standard C-code to program the \gls{GPU} (with a few extensions). The fact that parallel software and hardware was a small market at the time, the simplified use of the \gls{GPU} for parallel tasks opened up to many more customers. However, the main business is still graphics and the manufacturers can not make cards too expensive, especially at the cost of graphics performance (as integrating more double precision capacity would). This can be exemplified with the release of NVIDIA's Maxwell micro architecture, compared to the predecessor Kepler, both are similar but with Maxwell some of the double precision support was removed in favour of single precision (used in graphics).

Using GPUs in the context of data centers and \gls{HPC}, studies show that \gls{GPU} acceleration can reduce power\cite{huang2009energy} and its relevant to know the behaviour of the GPUs in the context of power and \gls{HPC}\cite{ghosh2012energy} for the best utilization.

\subsection{GPU vs CPU}

The \gls{GPU} is built on a principle of more execution units instead of higher clock-frequency to improve performance. Comparing the \gls{CPU} with the \gls{GPU}, the \gls{GPU} performs a much higher theoretical \gls{FLOPS} and at the same time at a better cost and energy efficiency\cite{owens2007survey}. The \gls{GPU} relies on using high memory bandwidth and fast context switching (run the next warp of threads) to compensate for lower frequency and hide memory latencies. The \gls{CPU} is excellent at sequential tasks with features like branch prediction that can not be found on the \gls{GPU}.

The \gls{GPU} thread is very lightweight and its creation have very little overhead, whereas on the \gls{CPU} the thread can be seen as an abstraction of the processor and switching a thread is considered expensive since the context have to be loaded each time. On the other hand, a \gls{GPU} is very inefficient if not enough threads are ready to work. Memory latencies are supposed to be hidden by switching in a new set of working threads as fast as possible.

A \gls{CPU} thread have its own registers but the \gls{GPU} thread work in groups where they share registers and memory. One can not give individual instructions to each thread, all of them will execute the same instruction. The figure \ref{fig:gpu-vs-cpu} demonstrates this by showing that by sharing control-structure and cache, the \gls{GPU} puts more resources on processing then the \gls{CPU} where more resources goes into control structures and memory cache.

\begin{figure}
	\centering
	\includestandalone[width={\textwidth}]{figures/gpu-cpu}
	\caption{The \gls{GPU} uses more transistors for data processing}
	\label{fig:gpu-vs-cpu}
\end{figure}

\section{Fast Fourier Transform}

This section extends the information from section \ref{sec:algorithms:fft} in the \textit{Benchmark application} chapter.

\subsection{\CTALG}

The Fast Fourier Transform is by far mostly associated with the {\CTALG} algorithm\cite{cooley1965algorithm}. The {\CTALG} algorithm is a devide and conquer algorithm that recursively breaks down a \gls{DFT} of any composite size of $N = N_1{\cdot}N_2$. The algorithm decomposes the \gls{DFT} into $s = \log_r{N}$ stages. The $N$-point \gls{DFT} is composed of $r$-point small \gls{DFT}s in $s$ stages. In this context the $r$-point \gls{DFT} is called radix-$r$ butterfly.

\subsubsection{Butterfly and radix-2}

The implementation of a $N$-point radix-2 \gls{FFT} algorithm have $\log_2{N}$ stages with $N/2$ butterfly operations per stage. A butterfly operation is an addition, a subtraction, followed by a multiplication by a twiddle factor, see figure \ref{fig:butterfly}.

\begin{figure}
	\centering
	% FFT Butterfly
\tikzstyle{n}= [circle, fill, minimum size=4pt,inner sep=0pt, outer sep=0pt]
\tikzstyle{mul} = [circle,draw,inner sep=-1pt, minimum size=7pt]
\tikzstyle{sub} = [circle,draw,inner sep=-1pt, minimum size=7pt]
\tikzstyle{add} = [circle,draw,inner sep=-1pt, minimum size=7pt]

\begin{tikzpicture}[
	yscale=1.3,
	xscale=1.8,
	node distance=0.7cm,
	auto]
    
	\node[n, pin={[pin edge={latex'-,black}]left:$x(0)$}] (N-0-0) at (0,-0) {};
	\node[n, pin={[pin edge={latex'-,black}]left:$x(1)$}] (N-0-1) at (0,-1) {};
                            
	\node[add, name=N-1-0, pin={[pin edge={-latex',black},pin distance=1.175cm]right:$X(0)$}] at (1,-0) {${+}$};
	\node[sub, name=N-1-1, label={[label distance=-10pt]90:${-}$}] at (1,-1) {};
            
	\node[mul, right of=N-1-1, label={above right:\tiny $W^{k}_{N}$},
		pin={[pin edge={-latex',black}]right:$X(1)$}] (N-2-1) {${\times}$};       

	\path (N-0-0) edge[-latex'] (N-1-0);	
	\path (N-0-1) edge[-latex'] (N-1-1);
	\path (N-1-1) edge[-latex'] (N-2-1);
      
    % Connect nodes
    \foreach \src / \dst in {	0/0, 0/1, 1/0, 1/1}
       \path (N-0-\src.east) edge[-latex'] (N-1-\dst.west);
       
\end{tikzpicture}
	\caption{Radix-2 butterfly operations}
	\label{fig:butterfly}
\end{figure}

Figure \ref{fig:cooley-tukey-8} shows an 8-point radix-2 \gls{DIF} \gls{FFT}. The input are in natural order whereas the output are in bit reversed order.

\begin{figure}
	\centering
	% FFT Butterfly
\tikzstyle{n}= [circle, fill, minimum size=4pt,inner sep=0pt, outer sep=0pt]
\tikzstyle{mul} = [circle,draw,inner sep=-1pt]

% Define two helper counters
\newcounter{x}\newcounter{y}
\begin{tikzpicture}[%
	%yscale=0.6,
	%xscale=1.2,
	node distance=0.25cm,
	auto]
    % Draw inputs
    \foreach \y in {0,...,7}
        \node[n, pin={[pin edge={latex'-,black}]left:$x(\y)$}] (N-0-\y) at (0,-\y) {};

	\foreach \y in {0,...,7}
        \foreach \x / \c in {1/1,2/3,3/4,4/6,5/7,6/9}
            \node[n, name=N-\c-\y] at (\x,-\y) {};           
    
    % Draw outputs
    \foreach \y in {0,...,7}
        \node[n, pin={[pin edge={-latex',black}]right:$X(\y)$}] (N-11-\y) at (8,-\y) {};
        
    % horizontal connections
    % Note the use of simple counter arithmetics to get correct
    % indexes.
    \foreach \y in {0,...,15}
    {
		\foreach \x in {0,1,3,4,7}
		{
			\setcounter{x}{\x}\stepcounter{x}
			\path (N-\x-\y) edge[-] (N-\arabic{x}-\y);
		}
	}
           
    % Draw the W_8 coefficients
    \setcounter{y}{0}
    \foreach \tmp in {0,1}
	{
    	\foreach \i in {0,...,3}
    	{
        	\path (N-5-\arabic{y}) edge[-] node {} (N-6-\arabic{y});
        	\addtocounter{y}{1}
    	}
    }

    % Draw the W_4 coefficients
    \setcounter{y}{0}
    \foreach \tmp in {0,1}
	{
    	\foreach \i in {0,1}
    	{
        	\path (N-5-\arabic{y}) edge[-] node {} (N-6-\arabic{y});
        	\addtocounter{y}{1}
    	}
    	\foreach \i in {0,1}
    	{
        	\path (N-5-\arabic{y}) edge[-] node {\tiny $W^{\i}_{8}$} (N-6-\arabic{y});
        	\addtocounter{y}{1}
    	}
    }
    
    % Connect nodes
    \foreach \sourcey / \desty in {	0/4,	1/5,	2/6,	3/7,
									4/0,	5/1,	6/2,	7/3}
        \path (N-3-\sourcey.east) edge[-] (N-4-\desty.west);
    \foreach \sourcey / \desty in {	0/0,	1/2,	2/0,	3/2,
    								0/1,	1/3,	2/1,	3/3,
                                   	4/4,	5/6,	6/4,	7/6,
                                   	4/5,	5/7,	6/5,	7/7,}
	{
        \path (N-6-\sourcey.east) edge[-] (N-7-\desty.west);
        \path (N-9-\sourcey.east) edge[-] (N-10-\desty.west);
    }
    
    % Add region boxes	
    % Complete stage
	\node[draw,densely dotted,fit=(N-0-0) (N-2-15),label=above:{stage 1}] {};
	\node[draw,densely dotted,fit=(N-3-0) (N-5-15),label=above:{stage 2}] {};
	\node[draw,fit=(N-6-0) (N-8-15),opacity=0,label=above:{stage 3},name=Stage-3] {};
	\node[draw,fit=(N-9-0) (N-\lastNode-15),opacity=0,label=above:{stage 4},name=Stage-4] {};
	\node[draw,fit=(N-11-0) (N-11-15),opacity=0,label=above:{output}] {};
	\node[draw,densely dotted,fit=(Stage-3) (Stage-4)] {};
	\node[draw,fit=(N-\lastNode-0) (N-11-15)] {};
\end{tikzpicture}
	\caption{8-point radix-2 \gls{FFT} using {\CTALG} algorithm}
	\label{fig:cooley-tukey-8}
\end{figure}

\subsubsection{Constant geometry}

Similar to {\CTALG} but with another data access pattern that uses the same indexing in all stages. This removes the overhead of calculating the data input index at each stage as in figure \ref{fig:cooley-tukey-8} where the top butterfly first stage require input $x[0], x[4]$ and in the second stage $x[0], x[2]$ whereas using the constant geometry algorithm in figure \ref{fig:constant-geometry-8} the equivalent input are $x[0], x[4]$ for all stages.

\begin{figure}
	\centering
	% FFT Butterfly
\tikzstyle{n}= [circle, fill, minimum size=4pt,inner sep=0pt, outer sep=0pt]
\tikzstyle{mul} = [circle,draw,inner sep=-1pt]

% Define two helper counters
\newcounter{x}\newcounter{y}
\begin{tikzpicture}[%
	yscale=0.6,
	xscale=1.2,
	node distance=0.25cm,
	auto]
    % The strategy is to create nodes with names: N-column-row
    % Input nodes are named N-0-0 ... N-0-15
    % Output nodes are named N-10-0 ... N-10-15

    % Draw inputs
    \foreach \y in {0,...,15}
        \node[n, pin={[pin edge={latex'-,black}]left:$x(\y)$}] (N-0-\y) at (0,-\y) {};
              
    % Draw outputs
    \foreach \y in {0,...,15}
        \node[n, pin={[pin edge={-latex',black}]right:$X(\y)$}] (N-11-\y) at (8,-\y) {};
              
   % draw connector nodes
    \foreach \y in {0,...,15}
        \foreach \x / \c in {1/1,2/3,3/4,4/6,5/7,6/9,7/10}
            \node[n, name=N-\c-\y] at (\x,-\y) {};
            
    % draw x nodes
    \foreach \y in {0,...,7}
        \foreach \x / \c  in {1/2}
            \node[mul, right of=N-\x-\y] (N-\c-\y) {};            
    \foreach \y in {8,...,15}
        \foreach \x / \c  in {1/2}
            \node[mul, right of=N-\x-\y] (N-\c-\y) {${\times}$};
    % 
    \foreach \y in {0,...,3}
        \foreach \x / \c  in {4/5}
            \node[mul, right of=N-\x-\y] (N-\c-\y) {};
    \foreach \y in {4,...,7}
        \foreach \x / \c  in {4/5}
            \node[mul, right of=N-\x-\y] (N-\c-\y) {${\times}$};
    \foreach \y in {8,...,11}
        \foreach \x / \c  in {4/5}
            \node[mul, right of=N-\x-\y] (N-\c-\y) {};
    \foreach \y in {12,...,15}
        \foreach \x / \c  in {4/5}
            \node[mul, right of=N-\x-\y] (N-\c-\y) {${\times}$};
    % 
    \foreach \y in {0,2,4,6,8,10,12,14}
        \foreach \x / \c  in {7/8}
            \node[mul, right of=N-\x-\y] (N-\c-\y) {};
    \foreach \y in {1,3,5,7,9,11,13,15}
        \foreach \x / \c  in {7/8}
            \node[mul, right of=N-\x-\y] (N-\c-\y) {${\times}$};    

    % horizontal connections
    % Note the use of simple counter arithmetics to get correct
    % indexes.
    \foreach \y in {0,...,15}
    {
		\foreach \x in {0,1,3,4,7}
		{
			\setcounter{x}{\x}\stepcounter{x}
			\path (N-\x-\y) edge[-] (N-\arabic{x}-\y);
		}
	}
       
    % Draw the W_16 coefficients
    \setcounter{y}{0}
    \foreach \i in {0,...,7}
    {
	   	\path (N-2-\arabic{y}) edge[-] node {} (N-3-\arabic{y});
	    \stepcounter{y}
    }
    \foreach \i in {0,...,7}
    {
    	\path (N-2-\arabic{y}) edge[-] node {\tiny $W^{\i}_{16}$} (N-3-\arabic{y});
        \stepcounter{y}
    }
    
    % Draw the W_8 coefficients
    \setcounter{y}{0}
    \foreach \tmp in {0,1}
	{
    	\foreach \i in {0,...,3}
    	{
        	\path (N-5-\arabic{y}) edge[-] node {} (N-6-\arabic{y});
        	\addtocounter{y}{1}
    	}
    	\foreach \i in {0,...,3}
    	{
        	\path (N-5-\arabic{y}) edge[-] node {\tiny $W^{\i}_{8}$} (N-6-\arabic{y});
        	\addtocounter{y}{1}
    	}
    }

    % Draw the W_4 coefficients
    \setcounter{y}{0}
	\foreach \tmp in {0,...,3}
	{    
		\foreach \i in {0,1}
		{
			\path (N-8-\arabic{y}) edge[-] node {} (N-9-\arabic{y});
			\stepcounter{y}
			\path (N-8-\arabic{y}) edge[-] node {\tiny $W^{\i}_{4}$} (N-9-\arabic{y});
			\stepcounter{y}
		}
    }
    
    % Connect nodes
    \foreach \sourcey / \desty in {	0/8,	1/9,	2/10,	3/11,
									4/12,	5/13,	6/14,	7/15,
									8/0,	9/1,	10/2,	11/3,
									12/4,	13/5,	14/6,	15/7}
       \path (N-0-\sourcey.east) edge[-] (N-1-\desty.west);
    \foreach \sourcey / \desty in {	0/4,	1/5,	2/6,	3/7,
									4/0,	5/1,	6/2,	7/3,
									8/12,	9/13,	10/14,	11/15,
									12/8,	13/9,	14/10,	15/11}
        \path (N-3-\sourcey.east) edge[-] (N-4-\desty.west);
    \foreach \sourcey / \desty in {	0/0,	1/2,	2/0,	3/2,
    								0/1,	1/3,	2/1,	3/3,
                                   	4/4,	5/6,	6/4,	7/6,
                                   	4/5,	5/7,	6/5,	7/7,
                                   	8/8,	9/10,	10/8,	11/10,
									8/9,	9/11,	10/9,	11/11,
									12/12,	13/14,	14/12,	15/14,
									12/13,	13/15,	14/13,	15/15}
	{
        \path (N-6-\sourcey.east) edge[-] (N-7-\desty.west);
        \path (N-9-\sourcey.east) edge[-] (N-10-\desty.west);
    }
    % Nodes are in bit-reverse order
    \foreach \sourcey / \desty in {	0/0,1/8,2/4,3/12,4/2,5/10,6,7/14,8/1,9,10/5,11/13,12/3,13/11,14/7,15/15}
	{
        \path (N-10-\sourcey.east) edge[-] (N-11-\desty.west);
    }
    
    % Add region boxes
	% Partial stage
	\def \lastNode {10}
	\node[draw,dashed,fit=(N-6-0) (N-\lastNode-3)] {};
	\node[draw,dashed,fit=(N-6-4) (N-\lastNode-7)] {};
	\node[draw,dashed,fit=(N-6-8) (N-\lastNode-11)] {};
	\node[draw,dashed,fit=(N-6-12) (N-\lastNode-15)] {};	
    % Complete stage
	\node[draw,densely dotted,fit=(N-0-0) (N-2-15),label=above:{stage 1}] {};
	\node[draw,densely dotted,fit=(N-3-0) (N-5-15),label=above:{stage 2}] {};
	\node[draw,fit=(N-6-0) (N-8-15),opacity=0,label=above:{stage 3},name=Stage-3] {};
	\node[draw,fit=(N-9-0) (N-\lastNode-15),opacity=0,label=above:{stage 4},name=Stage-4] {};
	\node[draw,fit=(N-11-0) (N-11-15),opacity=0,label=above:{output}] {};
	\node[draw,densely dotted,fit=(Stage-3) (Stage-4)] {};
	\node[draw,fit=(N-\lastNode-0) (N-11-15)] {};
\end{tikzpicture}
	\caption{Flow graph of an radix-2 \gls{FFT} using the constant geometry algorithm.}
	\label{fig:constant-geometry-8}
\end{figure}

\subsection{Parallelism in FFT}

By examining the \gls{FFT} algorithm, parallelism can be exploited in several ways. Naturally when decomposing the \gls{DFT} into radix-2 operations, parallelism can be achieved by mapping one thread per data input. That would however lead to unbalanced load as every second input is multiplied by the complex twiddle factor whereas the other is not. By selecting one thread per radix-2 butterfly operation instead, each thread will share the same workload.

\subsection{GPU algorithm}

The complete \gls{FFT} application can be implemented in two different kernels, one kernel executing over a single stage and one kernel executing the last stages that could fit within one block. The single-stage kernel, called \emph{global kernel}, would execute in sequential order each stage of the algorithm. Each execution would require in total as many threads as there are butterfly-operations. The host would supply the kernel with arguments depending on stage and problem size. See table \ref{tab:global-kernel} for full parameter list. The global kernel algorithm is shown in \ref{alg:device:global-kernel}. The global kernel would only be called for the number of stages not fitted in a single block (this depends on selected number of threads per block). The global kernel implements {\CTALG} algorithm.

\begin{table}
	\centering
	\begin{tabular}{|l|l|}
		\hline
		Parameter & Argument \\ \hline
		\textit{data} & Input/Output data buffer \\ \hline
		\textit{stage} & $[0,\log_{2}(N) - \log_{2}(N_{block})]$ \\ \hline
		\textit{bitmask} & $\Call{LeftShift}{\texttt{FFFFFFFF}_{16}, 32 - stage}$ \\ \hline
		\textit{angle} & $(2 \cdot \pi)/N$ \\ \hline
		\textit{dist} & $\Call{RightShift}{N, steps}$ \\ \hline		
	\end{tabular}
	\caption{Global kernel parameter list with argument depending on size of input $N$ and $stage$.}
	\label{tab:global-kernel}
\end{table}

\begin{algorithm}
	\centering
	\begin{algorithmic}[1]
		\Procedure{GlobalKernel}{$data, stage, bitmask, angle, dist$}
            \State $tid \gets \Call{GlobalThreadId}{}$ 
            \State $low \gets tid + (tid \And bitmask)$
            \State $high \gets low + dist$
            %\State // Calculate twiddle-factor
            \State $twMask \gets \Call{ShiftLeft}{dist - 1, stage}$
            \State $twStage \gets \Call{PowerOfTwo}{stage} \cdot tid$
            \State $a \gets angle \cdot (twStage \And twMask)$
            \State $\Call{Imag}{twiddleFactor} \gets \Call{Sin}{a}$
            \State $\Call{Real}{twiddleFactor} \gets \Call{Cos}{a}$
            %\State // Calculate butterfly-operations
            \State $temp \gets \Call{ComplexSub}{data_{low}, data_{high}}$
            \State $data_{low} \gets \Call{ComplexAdd}{data_{low}, data_{high}}$
            \State $data_{high} \gets \Call{ComplexMul}{temp, twiddleFactor}$
        \EndProcedure
	\end{algorithmic}
	\caption{Pseudo-code for the global kernel with input from the host.}
	\label{alg:device:global-kernel}
\end{algorithm}

\subsubsection{Shared/Local memory}

The \textit{local kernel} is always called and encapsulates all remaining stages and the bit reverse order output procedure. It is devised as to utilize shared memory completely for all stages. This reduces the primary memory access to read input and write output. The kernel implements the constant geometry algorithm to increase performance in the inner loop, the input and output index is calculated once. See algorithm \ref{alg:device:local-kernel}.

\begin{table}
	\centering
	\begin{tabular}{|l|l|}
		\hline
		Parameter & Argument \\ \hline
		\textit{in} & Input data buffer \\ \hline
		\textit{out} & Output data buffer \\ \hline
		\textit{angle} & $(2 \cdot \pi)/N$ \\ \hline
		\textit{stages} & $[\log_{2}(N) - \log_{2}(N_{block}), \log_{2}(N)]$ \\ \hline
		\textit{leadingBits} & $32 - \log_{2}(N)$ \\ \hline
		\textit{c} & Forward: $-1$, Inverse: $1/N$ \\ \hline
	\end{tabular}
	\caption{Local kernel parameter list with argument depending on size of input $N$ and number of stages left to complete.}
	\label{tab:local-kernel}
\end{table}

\begin{algorithm}
	\centering
	\begin{algorithmic}[1]
		\Procedure{LocalKernel}{$in$, $out$, $angle$, $stages$, $leadingBits$, $c$}
            \State let $shared$ be a shared/local memory buffer     
            \State $low  \gets \Call{ThreadId}{}$
            \State $high \gets low + \Call{BlockDim}{}$   
            \State $offset \gets \Call{BlockId}{} \cdot \Call{BlockDim}{} \cdot 2$
            \State $shared_{low}  \gets in_{low + offset}$
            \State $shared_{high} \gets in_{high + offset}$
            \State $\Call{ConstantGeometry}{shared, low, high, angle, stages}$
            \State $revLow  \gets \Call{BitReverse}{low + offset, leadingBits}$
            \State $revHigh \gets \Call{BitReverse}{high + offset, leadingBits}$
            \State $out_{revLow}  \gets \Call{ComplexMul}{c, shared_{low}}$
            \State $out_{revHigh} \gets \Call{ComplexMul}{c, shared_{high}}$
        \EndProcedure
        \Statex
        \Procedure{ConstantGeometry}{$shared$, $low$, $high$, $angle$, $stages$}
            \State $out_{i} \gets low \cdot 2$
            \State $out_{ii} \gets outI + 1$
            \For {$stage \gets 0, stages - 1$}
            	\State $bitmask \gets \Call{ShiftLeft}{0xFFFFFFFF, stage}$
            	\State $a \gets angle \cdot (low \And bitmask)$
            	\State $\Call{Imag}{twiddleFactor} \gets \Call{Sin}{a}$
            	\State $\Call{Real}{twiddleFactor} \gets \Call{Cos}{a}$
				\State $temp \gets \Call{ComplexSub}{shared_{low}, shared_{high}}$
				\State $shared_{out_{i}} \gets \Call{ComplexAdd}{shared_{low}, shared_{high}}$
				\State $shared_{out_{ii}} \gets \Call{ComplexMul}{twiddleFactor, temp}$
			\EndFor
        \EndProcedure
	\end{algorithmic}
	\caption{Pseudo-code for the local kernel with input from the host.}
	\label{alg:device:local-kernel}
\end{algorithm}

\subsubsection{Register width}

An important \gls{GPU} specific detail is that all fast integer arithmetic is based on $32$ bit registers. Procedures using bitwise operations uses this architectural specific information as in the bitmask parameter in table \ref{tab:global-kernel} and leadingBits parameter in table \ref{tab:local-kernel}. The bitmask parameter is used to get the offset for each stage using the {\CTALG} algorithm. The leadingBits parameter is used when performing a bit-reverse operation and the leading zeroes resulting of using a $32$ bit register needs to be removed.

Bit-reverse example: if the total size is $1024$ elements then the last $\log_{2}(1024) = 10$ bits are used. When using $1008 = 1111110000_{2}$ for bit-reversal in this context with the problem size $1024$, the result is $63$. Using a $32$ bit register:
\begin{equation}
	1008 = 00000000000000000000001111110000_{2}
\end{equation}
bits reversed:
\begin{equation}
	264241152 = 00001111110000000000000000000000_{2} 
\end{equation}
The leading zeroes becomes trailing zeroes that need to be removed. A logic right shift operation by the length of leadingBits = $32 - \log_{2}(1024) = 22$ solves this.

\section{Related research}

The main latest scientific interest have mainly been targeting {\CU} and {\OCL} for comparisons. Benchmarking between the two have established that there is difference in {\CU}s favour, however it can be due to unfair comparisons\cite{fang2011comprehensive} and with the correct tuning {\OCL} can be just as fast. The same paper stated that the biggest difference came from running the forward \gls{FFT} algorithm. Examination showed that large differences could be found in the \gls{PTX} instructions (intermediary \gls{GPU} code).

Porting from {\CU} to {\OCL} without loosing performance have been explored in \cite{du2012cuda} where the goal was to achieve a performance-portable solution, some of the main differences between the technologies are described in that paper.