% FFT Butterfly
\tikzstyle{n}= [circle, fill, minimum size=4pt,inner sep=0pt, outer sep=0pt]
\tikzstyle{mul} = [circle,draw,inner sep=-1pt]

% Define two helper counters
\newcounter{x}\newcounter{y}
\begin{tikzpicture}[%
	yscale=0.5,
	xscale=1.2,
	node distance=0.3cm,
	auto]
    % The strategy is to create nodes with names: N-column-row
    % Input nodes are named N-0-0 ... N-0-15
    % Output nodes are named N-10-0 ... N-10-15

    % Draw inputs
    \foreach \y in {0,...,15}
        \node[n, pin={[pin edge={latex'-,black}]left:$x(\y)$}] (N-0-\y) at (0,-\y) {};
              
    % Draw outputs
    \foreach \y / \idx in {0,...,15}
        \node[n, pin={[pin edge={-latex',black}]right:$X(\y)$}] (N-11-\y) at (8,-\y) {};
              
   % draw connector nodes
    \foreach \y in {0,...,15}
        \foreach \x / \c in {1/1,2/3,3/4,4/6,5/7,6/9,7/10}
            \node[n, name=N-\c-\y] at (\x,-\y) {};
            
    % draw x nodes
    \foreach \y in {0,...,7}
        \foreach \x / \c  in {1/2}
            \node[mul, right of=N-\x-\y] (N-\c-\y) {};            
    \foreach \y in {8,...,15}
        \foreach \x / \c  in {1/2}
            \node[mul, right of=N-\x-\y] (N-\c-\y) {${\times}$};
    % 
    \foreach \y in {0,...,3}
        \foreach \x / \c  in {4/5}
            \node[mul, right of=N-\x-\y] (N-\c-\y) {};
    \foreach \y in {4,...,7}
        \foreach \x / \c  in {4/5}
            \node[mul, right of=N-\x-\y] (N-\c-\y) {${\times}$};
    \foreach \y in {8,...,11}
        \foreach \x / \c  in {4/5}
            \node[mul, right of=N-\x-\y] (N-\c-\y) {};
    \foreach \y in {12,...,15}
        \foreach \x / \c  in {4/5}
            \node[mul, right of=N-\x-\y] (N-\c-\y) {${\times}$};
    % 
    \foreach \y in {0,2,4,6,8,10,12,14}
        \foreach \x / \c  in {7/8}
            \node[mul, right of=N-\x-\y] (N-\c-\y) {};
    \foreach \y in {1,3,5,7,9,11,13,15}
        \foreach \x / \c  in {7/8}
            \node[mul, right of=N-\x-\y] (N-\c-\y) {${\times}$};    

    % horizontal connections
    % Note the use of simple counter arithmetics to get correct
    % indexes.
    \foreach \y in {0,...,15}
    {
		\foreach \x in {0,1,3,4,7}
		{
			\setcounter{x}{\x}\stepcounter{x}
			\path (N-\x-\y) edge[-] (N-\arabic{x}-\y);
		}
	}
       
    % Draw the W_16 coefficients
    \setcounter{y}{0}
    \foreach \i in {0,...,7}
    {
	   	\path (N-2-\arabic{y}) edge[-] node {} (N-3-\arabic{y});
	    \stepcounter{y}
    }
    \foreach \i in {0,...,7}
    {
    	\path (N-2-\arabic{y}) edge[-] node {\tiny $W^{\i}_{16}$} (N-3-\arabic{y});
        \stepcounter{y}
    }
    
    % Draw the W_8 coefficients
    \setcounter{y}{0}
    \foreach \tmp in {0,1}
	{
    	\foreach \i in {0,...,3}
    	{
        	\path (N-5-\arabic{y}) edge[-] node {} (N-6-\arabic{y});
        	\addtocounter{y}{1}
    	}
    	\foreach \i in {0,...,3}
    	{
        	\path (N-5-\arabic{y}) edge[-] node {\tiny $W^{\i}_{8}$} (N-6-\arabic{y});
        	\addtocounter{y}{1}
    	}
    }

    % Draw the W_4 coefficients
    \setcounter{y}{0}
	\foreach \tmp in {0,...,3}
	{    
		\foreach \i in {0,1}
		{
			\path (N-8-\arabic{y}) edge[-] node {} (N-9-\arabic{y});
			\stepcounter{y}
			\path (N-8-\arabic{y}) edge[-] node {\tiny $W^{\i}_{4}$} (N-9-\arabic{y});
			\stepcounter{y}
		}
    }
    
    % Connect nodes
    \foreach \sourcey / \desty in {	0/8,	1/9,	2/10,	3/11,
									4/12,	5/13,	6/14,	7/15,
									8/0,	9/1,	10/2,	11/3,
									12/4,	13/5,	14/6,	15/7}
       \path (N-0-\sourcey.east) edge[-] (N-1-\desty.west);
    \foreach \sourcey / \desty in {	0/4,	1/5,	2/6,	3/7,
									4/0,	5/1,	6/2,	7/3,
									8/12,	9/13,	10/14,	11/15,
									12/8,	13/9,	14/10,	15/11}
        \path (N-3-\sourcey.east) edge[-] (N-4-\desty.west);
    \foreach \sourcey / \desty in {	0/0,	1/2,	2/0,	3/2,
    								0/1,	1/3,	2/1,	3/3,
                                   	4/4,	5/6,	6/4,	7/6,
                                   	4/5,	5/7,	6/5,	7/7,
                                   	8/8,	9/10,	10/8,	11/10,
									8/9,	9/11,	10/9,	11/11,
									12/12,	13/14,	14/12,	15/14,
									12/13,	13/15,	14/13,	15/15}
	{
        \path (N-6-\sourcey.east) edge[-] (N-7-\desty.west);
        \path (N-9-\sourcey.east) edge[-] (N-10-\desty.west);
    }
    % Nodes are in bit-reverse order
    \foreach \sourcey / \desty in {	0/0,1/8,2/4,3/12,4/2,5/10,6,7/14,8/1,9,10/5,11/13,12/3,13/11,14/7,15/15}
	{
        \path (N-10-\sourcey.east) edge[-] (N-11-\desty.west);
    }
    
    % Add region boxes
	% Partial stage
	\def \lastNode {10}
	\node[draw,dashed,fit=(N-6-0) (N-\lastNode-3)] {};
	\node[draw,dashed,fit=(N-6-4) (N-\lastNode-7)] {};
	\node[draw,dashed,fit=(N-6-8) (N-\lastNode-11)] {};
	\node[draw,dashed,fit=(N-6-12) (N-\lastNode-15)] {};	
    % Complete stage
	\node[draw,densely dotted,fit=(N-0-0) (N-2-15),label=above:{stage 1}] {};
	\node[draw,densely dotted,fit=(N-3-0) (N-5-15),label=above:{stage 2}] {};
	\node[draw,fit=(N-6-0) (N-8-15),opacity=0,label=above:{stage 3},name=Stage-3] {};
	\node[draw,fit=(N-9-0) (N-\lastNode-15),opacity=0,label=above:{stage 4},name=Stage-4] {};
	\node[draw,fit=(N-11-0) (N-11-15),opacity=0,label=above:{output}] {};
	\node[draw,densely dotted,fit=(Stage-3) (Stage-4)] {};
	\node[draw,fit=(N-\lastNode-0) (N-11-15)] {};
\end{tikzpicture}