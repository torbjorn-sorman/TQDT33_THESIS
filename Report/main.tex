\documentclass[
	%draft,
	utf8,%     More capable input encoding than latin-1.
	parskip,%  For vertical whitespace between paragraphs.  This comes down to more than just using parskip.sty, so it's better to use this class option.
	% S5MP % If you intend to really use margin paragraphs (not recommended!).
	%  crop,%     Produce output with crop marks and paper size A4.  Liu-Tryck should like this.  Automatically adds information, including the physical page number, at the top of each page.
	%     Add option 'noInfo' to suppress the info at the top of each page when using option 'crop'.
	% Font options: 'kp' (default), 'times', 'lm'.  The KpFonts (loaded using 'kp'), is the most complete font among the provided options.  Among other, it supports slanted small caps.  See rtthesis.cls for more details regarding the font options.
	largesmallcaps,intlimits,widermath,% Good options to KpFonts.
	sharecounter,nobreak,definition=marks,%  See comments in the results chapter of this document for more information on these options!
	%article,
	numbers, % If you want to cite references by numbers, use this option.
	noparts % Use option 'noparts' if you do not make use of part divisions.	
]{rtthesis}
\usepackage{etex}

\usepackage{graphicx}
\usepackage{standalone}
\usepackage{mythesis}
\usepackage{multirow}
\usepackage{listings}
\usepackage{framed}
\usepackage{todo}
\usepackage{pgfplots}
\usepackage{tablefootnote}
\usepackage{algorithm}
\usepackage{algpseudocode}
\usepackage{mathtools}
\DeclarePairedDelimiter{\ceil}{\lceil}{\rceil}

\usetikzlibrary{
	fit,
	positioning,
	arrows,
	arrows.meta,
	patterns
}
\usepgfplotslibrary{units}

\newcommand{\CTALG}{Cooley-Tukey}
\newcommand{\CGALG}{constant geometry}

\newcommand{\NVCARD}{GeForce GTX 670}
\newcommand{\AMDCARD}{Radeon R7 R260X}
\newcommand{\INTELCPU}{Intel Core i7 3770K 3.5GHz CPU}

\newcommand{\CU}{CUDA}
\newcommand{\OCL}{OpenCL}
\newcommand{\GL}{OpenGL}
\newcommand{\DX}{DirectCompute}
\newcommand{\OMP}{OpenMP}
\newcommand{\CPP}{C/C++}
\newcommand{\CUFFT}{cuFFT}
\newcommand{\CLFFT}{clFFT}
\newcommand{\FFTW}{FFTW}

\newcommand\code{\texttt}

\begin{document}
%
\selectlanguage{english}
\makeFrontPage
\frontmatter
\maketitle
\makeLibraryPage{The computational capacity of graphics cards for general-purpose computing have progressed fast over the last decade. A major reason is computational heavy computer games, where standard of performance and high quality graphics constantly rise. Another reason are better suitable technologies for programming the graphics cards. Combined, the product is high raw performance devices and means to access that performance. This thesis investigates some of the current technologies for general-purpose computing on graphics processing units. Technologies are primarily compared by means of benchmarking performance and secondarily by factors concerning programming and implementation. The choice of technology can have a large impact on performance. The benchmark application found the difference of the fastest technology, CUDA, compared to the slowest, OpenCL, to be twice as long execution time. The benchmark application also found out that the older technologies, OpenGL and DirectX, are competitive with CUDA and OpenCL in terms of resulting raw performance.%}
\begin{abstract}[english]
	The computational capacity of graphics cards for general-purpose computing have progressed fast over the last decade. A major reason is computational heavy computer games, where standard of performance and high quality graphics constantly rise. Another reason are better suitable technologies for programming the graphics cards. Combined, the product is high raw performance devices and means to access that performance. This thesis investigates some of the current technologies for general-purpose computing on graphics processing units. Technologies are primarily compared by means of benchmarking performance and secondarily by factors concerning programming and implementation. The choice of technology can have a large impact on performance. The benchmark application found the difference of the fastest technology, CUDA, compared to the slowest, OpenCL, to be twice as long execution time. The benchmark application also found out that the older technologies, OpenGL and DirectX, are competitive with CUDA and OpenCL in terms of resulting raw performance.%
\end{abstract}
\begin{acknowledgments}
	I would like to thank Åsa for the opportunity to make this thesis work at MindRoad AB. I would also like to thank Ingemar and Robert at ISY.

	\addvspace{1em}
	\begin{flushright}
    \textit{%
		Linköping, January 2015\\
		Torbjörn Sörman%
	}
	\end{flushright}
\end{acknowledgments}


\tableofcontents
\begin{notation}% Passing the option "old" to the notation environment will redefine the notationtabular environment so that it produces an old style LaTeX tabular instead of a ctable.sty style tabular.
	\centering
	
	\begin{notationtabular}{Glossary}{Term}{Meaning}
		Kernel & A GPU program \\
		Thread & A way for a program to split itself to many simultaneously running tasks \\
		Block & A way to organize threads, a block is executed by a multiprocessing unit \\
	\end{notationtabular}

	\begin{notationtabular}{Abbreviations}{Abbreviation}{Meaning}
		1D & One-dimensional \\
		2D & Two-dimensional \\
		API & Application Programming Interface \\
		CPU & Central Processing Unit \\
		CUDA & Compute Unified Device Architecture, a technology developed by NVIDIA used to program their GPUs \\
		DFT & Discrete Fourier Transform \\
		DIF & Decimation In Frequency \\
		DIT & Decimation In Time \\
		DRAM & Dynamic Random Access Memory \\
		DWT & Discrete Wavelet Transform \\
		FFT & Fast Fourier Transform \\
		GPGPU & General-Purpose computing on Graphics Processing Units \\
		GPU & Graphics Processing Unit \\
		OpenCL & Open Computing Language, initially developed by Apple, today the Khronos Group \\
		SIMD & Single Instruction, Multiple Data \\
		SIMT & Single Instruction, Multiple Threads \\
		SM & Streaming multiprocessor, example: NVIDIA’s GPUs contains many streaming multiprocessors and each consists of scalar processors running threads parallel\\   
		
	\end{notationtabular}
	
\end{notation}


\mainmatter
\chapter{Introduction}\label{cha:intro}
This chapter gives an introduction to the thesis. The background, purpose and goal of the thesis, describes a list of abbreviations and the structure of this report.

\section{Background}
Technical improvements of hardware has for a long period of time been the best way to solve computationally demanding problems faster. However, during the last decades, the limit of what can be achieved by hardware improvements appear to have been reached: The operating frequency of the \gls{CPU} does no longer significantly improve. Problems relying on single thread performance are limited by three primary technical factors:
\begin{enumerate}
	\item The \gls{ILP} wall
	\item The memory wall
	\item The power wall
\end{enumerate}

The first wall states that it is hard to further exploit simultaneous \gls{CPU} instructions: techniques such as instruction pipelining, superscalar execution and \gls{VLIW} exists but complexity and latency of hardware reduces the benefits.

The second wall, the gap between \gls{CPU} speed and memory access time, that may cost several hundreds of \gls{CPU} cycles if accessing primary memory.

The third wall is the power and heating problem. The power consumed is increased exponentially with each factorial increase of operating frequency.

Improvements can be found in exploiting parallelism. Either the problem itself is already inherently parallelizable, or reconstruct the problem. This trend manifests in development towards use and construction of multi-core microprocessors. The \gls{GPU} is one such device, it originally exploited the inherent parallelism within visual rendering but now is available as a tool for massively parallelizable problems.

\section{Problem statement}
Programmers might experience a threshold and a slow learning curve to move from a sequential to a thread-parallel programming paradigm that is \gls{GPU} programming. Obstacles involve learning about the hardware architecture, and restructure the application. Knowing the limitations and benefits might even provide reason to not utilize the \gls{GPU}, and instead choose to work with a multi-core \gls{CPU}.

Depending on ones preferences, needs, and future goals; selecting one technology over the other can be very crucial for productivity. Factors concerning productivity can be portability, hardware requirements, programmability, how well it integrates with other frameworks and \gls{API}s, or how well it is supported by the provider and developer community. Within the range of this thesis, the covered technologies are \gls{CUDA}, \gls{OpenCL}, DirectCompute (\gls{API} within DirectX), and \gls{OpenGL} Compute Shaders.

\section{Purpose and goal of the thesis work}
The purpose of this thesis is to evaluate, select, and implement an application suitable for \gls{GPGPU}.

To implement the same application in technologies for \gls{GPGPU}: ({\CU}, {\OCL}, {\DX}, and {\GL}), compare \gls{GPU} results with results from an sequential {\CPP} implementation and an multi-core {\OMP} implementation, and to compare the different technologies by means of benchmarking, and the goal is to make qualitative assessments of how it is to use the technologies.

\section{Delimitations}
Only one benchmark application algorithm will be selected, the scope and time required only allows for one algorithm to be tested. Each technology have different debugging and profiling tools and those are not included in the comparison of the technologies. However important such tool can be, they are of a subjective nature and harder to put a measure on.
\chapter{Benchmark algorithm}\label{cha:algorithms}
This part cover possible applications for a \gls{GPGPU} study. The basic theory and motivation why they are suitable for benchmarking \gls{GPGPU} technologies is presented.% The \gls{FFT} algorithm is selected for the benchmarking application in this thesis work and will be in more detail in the next chapter.

\section{Discrete Fourier Transform}
The Fourier transform is of use when analysing the spectrum of a continuous analogue signal. When applying transformation to a signal it is decomposed into the frequencies that makes it up. In digital signal analysis the \gls{DFT} is the counterpart of the Fourier transform for analogue signals. The \gls{DFT} converts a sequence of finite length into a list of coefficients of a finite combination of complex sinusoids. Given that the sequence is a sampled function from the time or spatial domain it's a conversion to the frequency domain. It is defined as 
\begin{equation}
X_k=\sum_{n=0}^{N-1}x(n)W_N^{kn}, k \in {[0, N-1]}
\end{equation}
where $W_N=e^{-\frac{i2{\pi}}{N}}$, commonly named the twiddle factor\cite{gentleman1966fast}.

The \gls{DFT} is used in many practical applications to perform Fourier analysis. Its a powerful mathematical tool that enables a perspective from another domain where difficult and complex problems becomes easier to analyse. Practically used in digital signal processing, such as discrete samples of sound waves, radio signal or any continuous signal over a finite time interval. In image processing the sampled sequence can be pixels along a row or column. The \gls{DFT} takes input in complex numbers and outputs in complex coefficients. In practical applications the input is usually real numbers.

\subsection{Fast Fourier Transform}\label{sec:algorithms:fft}
The problem with the \gls{DFT} is that the direct computation require $\mathcal{O}(n^n)$ complex multiplications and complex additions, that makes it computationally heavy and impractical in high throughput applications. The \gls{FFT} is one of the most common algorithm used to compute the \gls{DFT} of a sequence. An \gls{FFT} computes the transformation by factorizing the transformation matrix of the \gls{DFT} into a product of mostly zero factors. This reduces the order of computations to $\mathcal{O}(n\log{}n)$ complex multiplications and additions.

The FFT was made popular in 1965\cite{cooley1965algorithm} by J.W Cooley and John Tukey and it found its way into practical use at the same time and meant a serious breakthrough in digital signal processing \cite{cooley1969fast, brigham1967fast}. However the complete algorithm was not invented at the time, the history of the Cooley-Tukey \gls{FFT} algorithm can be traced back to around 1805 by work of the famous mathematician Carl Friedrich Gauss\cite{heideman1984gauss}. The algorithm is a divide and conquer algorithm that relies on recursively dividing the input into sub-blocks and eventually the problem is small enough to be solved and the sub-blocks are combined into the final result.
\section{Image processing}
Image processing consists of a wide range of domains. Earlier academic work with performance evaluation on the \gls{GPU}\cite{park2011design} tested four major domains (\gls{3D} shape reconstruction, feature extraction, image compression and computational photography) and compared with the \gls{CPU}. Image processing is often by nature parallel and one can expect good results on a \gls{GPU}.

Most of image processing algorithms apply the same computation on a number of pixels and that is a typically data parallel operation. Some algorithms can then be expected to have huge speed up compared to an efficient \gls{CPU} implementation. A representative task is applying a simple image filter that gathers neighboring pixel-values and compute a new value for a pixel. If done with respect to the underlying structure of the system one can expect a speedup near linear to the number of computational cores used. That is a \gls{CPU} with four cores can theoretically expect a near four time speedup compared to a single core. This extends to a \gls{GPU} so a \gls{GPU} with n cores can in ideal cases expect a speedup in the order of n. An example of this is a Gaussian blur (or smoothing) filter.

\section{Image compression}
The image compression standard \emph{JPEG2000} offers algorithms with parallelism but is very computationally and memory intensive. The standard aims to improve performance over JPEG but also adding new features. The following sections are part of the JPEG2000 algorithm\cite{christopoulos2000jpeg2000}
\begin{enumerate}
	\item Color Component transformation
	\item Tiling
	\item Wavelet transform
	\item Quantization
	\item Coding
\end{enumerate}

The computation heavy parts can be identified as the \gls{DWT} and the encoding engine using \gls{EBCOT} Tier-1.

The important difference between the older format \emph{JPEG} compared to JPEG2000 is the use of \gls{DWT} instead of \gls{DCT}. In comparison to the \gls{DFT}, the \gls{DCT} operates solely on real values. \gls{DWT}s on the other hand uses another representation that allows for a time complexity of $\mathcal{O}(N)$.

\section{Linear algebra}
Linear algebra is central to both pure and applied mathematics. In scientific computing it's a highly relevant problem to solve dense linear systems efficiently. In the initial uses of GPUs in scientific computing, the graphics pipeline was successfully used for linear algebra through programmable vertex and pixel shaders \cite{kruger2003linear}. Methods and systems used later on for utilizing \gls{GPU}s have been shown efficient also in hybrid system (multi-core \gls{CPU}s + \gls{GPU}s)\cite{tomov2010dense}. Linear algebra is highly suitable for \gls{GPU}s and with careful calibration it is possible to reach 80\%-90\% of the theoretical peak speed of large matrices\cite{volkov2008benchmarking}.

Common operations are vector addition, scalar multiplication, dot products, linear combinations, and matrix multiplication. Matrix multiplications are of much interest since the high time complexity $\mathcal{O}(N^3)$ makes it a bottleneck in many algorithms. Matrix decomposition like LU, QR and Cholesky decomposition are used very often and are subject for benchmark applications targeting \gls{GPU}s\cite{volkov2008benchmarking}.

\section{Sorting}
The sort operation is an important part in computer science and have been a classic problem to work on. There exists several sorting techniques and depending on problem and requirements, a suitable algorithm is found by examining the attributes.

Sorting algorithms can be organized into two categories, data-driven and data-independent. The classic quicksort algorithm is probably the best known example of a data-driven sorting algorithm. It performs with time complexity $\mathcal{O}(n\log{n})$ on average but have a time complexity of $\mathcal{O}(n^2)$ in the worst case. Another data-driven algorithm that does not have this problem is heap sort but instead it suffers from difficult data access patterns. Data-driven algorithms are not the easiest to parallelize since the behaviour is unknown and may cause bad load balancing.

The other category are the algorithms that always perform the same process no matter what the data. This behaviour makes suitable for implementation on multiple processors, fixed sequences of instructions where the moment in which data is synchronized and communication must occur are known in advance.

\subsection{Efficient sorting}
Bitonic sort have been used early on in the utilization of \gls{GPU}s for sorting, even though it has the time complexity of $\mathcal{O}(n\log{n^2})$ it's been an easy way of doing reasonably efficient sort on \gls{GPU}s. Other high performance sorting on \gls{GPU}s are often combinations of algorithms. Examples of fast sorting algorithms on GPUs have used bucket sort or quicksort that first splits the list into sub list and then sort in parallel with merge sort or by using bitonic sort followed by merge sort.

A popular algorithm for GPUs have been variants of radix sort which is a non-comparative integer sorting algorithm. Radix sorts can be described as being easy to implement and still as efficient as more sophisticated algorithms. Radix sort works by grouping the integer keys by the individual digits value in the same significant position and value.

\section{Criteria for Algorithm Selection}
A benchmarking application is sought after that have the necessary complexity and relevance to both practical uses and the scientific community. The algorithm with enough complexity and challenges is the \gls{FFT}, compared to the other presented algorithms the \gls{FFT} are more complex than the matrix operations and the regular sorting algorithms. The \gls{FFT} does not demand as much domain knowledge as the image compression algorithms but its still a very important algorithm for many applications.

The difficulties working with multi-core systems are applied to \gls{GPU}s. What \gls{GPU}s are missing compared to multi-core \gls{CPU}s are the power of working in sequential, instead \gls{GPU}s are excellent at fast context switching and hiding memory latencies. Most effort of working with \gls{GPU}s extends to supply tasks with enough parallelism, avoiding branching and refine memory access patterns. One important issue is also the host to device memory transfer-time. If the algorithm is much faster on the \gls{GPU}, a \gls{CPU} could still be faster if the host to device and back transfer is a large part of the total time.

By selecting an algorithm that have much scientific interest and history; relevant comparisons can be made and it is sufficient to say that one can demand a reasonable performance by utilizing information sources using similar implementations on \gls{GPU}s.
\chapter{Theory}

This chapter will give an introduction to the FFT algorithm and a brief introduction of the Graphics Processing Unit (GPU).

\section{Graphics Processing Unit}

A GPU is traditionally specialized hardware for efficient manipulation of computer graphics and image processing. The inherent parallel structure of images and graphics makes them very efficient at some more general problems where parallelism can be exploited. The concept of General-purpose computing on graphics processing units (GPGPU) is applying a problem to the GPU platform instead of the CPU or multi-core system.

\subsection{Graphics hardware pipeline}

The traditional GPU can be described as a pipeline of a few linked steps. Every step receives input from the previous, processes the input and send it to the next. 

\subsection{GPGPU}

In the early days of GPGPU one had to rely on knowing a lot of graphics abstractions since the then available APIs was created with graphic processing in mind. The dominant APIs was OpenGL and DirectX (Direct3D).

\subsection{GPU vs CPU}

The GPU is build on a principle of more execution units instead of higher clock-frequency to improve performance. Comparing the two and the GPU performs a much higher floating point operations per second (FLOP) if running at optimal conditions. What the GPU sacrifice is the ability to run one task in sequential. The GPU relies much on using high memory bandwidth and fast context switching (run the next warp of threads) to compensate for lower frequency. The CPU is excellent at sequential tasks and uses branch prediction and ... among other finesses that is not used on the GPU.

\section{Fast Fourier Transform}

The Fast Fourier Transform is by far mostly associated with the Cooley-Tukey algorithm.

\subsection{FFT Cooley-Tukey}

The Cooley-Tukey algorithm is a devide and conquer algorithm that recursively breaks down a DFT of any composite size of $N = N_1{\cdot}N_2$.  

"
This is a divide and conquer algorithm that recursively breaks down a DFT of any composite size N = N1N2 into many smaller DFTs of sizes N1 and N2, along with O(N) multiplications by complex roots of unity traditionally called twiddle factors (after Gentleman and Sande, 1966[11]).
"

\subsubsection{FFT Constant Geometry}

Essentially the same algorithm but with some clever indexing the structure of the FFT do not changes through the stages and the indexing stays the same.

\subsection{FFT parallelism}

By looking at the FFT algorithm illustrated, it is easy to see how one can split operations over parallel tasks. Naturally one can start by selecting one thread per data input, however that would lead to unbalanced load as the second input is multiplied by the twiddle factor. By selecting one thread per butterfly operation each thread will share the same arithmetic workload.

\subsubsection{Problems parallelizing}

Memory latencies can be hidden by caches and fast context switching, however the distance in memory will matter. Large distances between data in the butterfly operations will make the memory the bottleneck. Coalesced memory read goes well with good context switching since the fewer memory request will be performed.

\subsection{FFT on GPU}

\begin{algorithm}
	\centering
	\begin{algorithmic}[1]
		\Procedure{GlobalKernel}{$data, bitmask, angle, stage, dist$}
            \State $tid \gets \Call{GlobalThreadId}{}$
            \newline
            \State // Calculate input offset          
            \State $low \gets tid + (tid \And bitmask)$
            \State $high \gets low + dist$            
            \newline
            \State // Calculate twiddle-factor
            \State $angle \gets angle \cdot ((tid \cdot 2^{stage}) \And \Call{ShiftLeft}{dist - 1, stage})$
            \State $\Call{Imag}{twiddleFactor} \gets \Call{Sin}{angle}$
            \State $\Call{Real}{twiddleFactor} \gets \Call{Cos}{angle}$
            \newline
            \State // Calculate butterfly-operations
            \State $temp \gets \Call{ComplexSub}{data_{low}, data_{high}}$
            \State $data_{low} \gets \Call{ComplexAdd}{data_{low}, data_{high}}$
            \State $data_{high} \gets \Call{ComplexMul}{temp, twiddleFactor}$
        \EndProcedure
	\end{algorithmic}
	\caption{Pseudo-code for the global kernel with input from the host.}
	\label{alg:device:global-kernel}
\end{algorithm}

\begin{algorithm}
	\centering
	\begin{algorithmic}[1]
		\Procedure{LocalKernel}{$input$, $output$, $angle$, $stepsLeft$, $leadingBits$, $scalar$, $blockRange$}
            \State let $shared$ be shared/local memory
            \State $offset \gets blockIdx.x \cdot blockDim.x \cdot 2$
            \State // Calculate input offset          
            \State $low \gets tid + (tid \And bitmask)$
            \State $high \gets low + dist$            
            \newline
            \State // Calculate twiddle-factor
            \State $angle \gets angle \cdot ((tid \cdot 2^{stage}) \And \Call{ShiftLeft}{dist - 1, stage})$
            \State $\Call{Imag}{twiddleFactor} \gets \Call{Sin}{angle}$
            \State $\Call{Real}{twiddleFactor} \gets \Call{Cos}{angle}$
            \newline
            \State // Calculate butterfly-operations
            \State $temp \gets \Call{ComplexSub}{data_{low}, data_{high}}$
            \State $data_{low} \gets \Call{ComplexAdd}{data_{low}, data_{high}}$
            \State $data_{high} \gets \Call{ComplexMul}{temp, twiddleFactor}$
        \EndProcedure
	\end{algorithmic}
	\caption{Pseudo-code for the local kernel with input from the host.}
	\label{alg:device:local-kernel}
\end{algorithm}
\chapter{Technologies}

Five different multi-core technologies are used in this study. One is specialized in GPGPU, namely CUDA. OpenGL and DirectCompute are parts of graphic programming languages but breaks away from the graphics abstraction with \textit{Compute Shaders}. OpenCL aims at any heterogeneous multi-core system and is used in this study to use on the GPU. To compare with the CPU, OpenMP is included as a fast and easy way to parallelize C/C++ -code.

\section{CUDA}

CUDA is an acronym for Compute Unified Device Architecture, developed by NVIDIA and released in 2006. CUDA is an extension of the C/C++ language and have its own compiler. CUDA supports the functionality to execute kernels, modify the graphic card RAM memory and the use of several optimized function libraries such as \textit{cuBLAS} (CUDA implementation of BLAS, Basic Linear Algebra Subprograms) or \textit{cuFFT} (CUDA implementation of FFT).

A program running on the GPU is called a kernel. The GPU is referred to as the \textit{device} and the the CPU is called the \textit{host}. To run a CUDA kernel all that is needed is to declare the program with a function type specifier, see table \ref{tab:cuda:func-types}, and call it from the host with launch parameters.

\begin{table}
	%\includestandalone[width=\textwidth]{tables/cuda-function-types}
	\centering
	\input{tables/cuda-function-types}
	\caption{Table of function types in CUDA.}
	\label{tab:cuda:func-types}
\end{table}

\begin{figure}
	\centering
	\lstset{language=C++}
	\begin{framed}
	\begin{lstlisting}
// Device program (GPU)
__global__ void myKernel(float val)
{
  int tid = threadIdx.x + blockDim.x * blockIdx.x;
}

// Host program (CPU)
__host__ void myFunction(float value)
{
  dim3 threads = {1024, 1, 1};
  dim3 blocks  = {1, 1, 1};
  myKernel<<<blocks, threads>>>(value);
}
	\end{lstlisting}
	\end{framed}
\end{figure}

\section{OpenCL}

\section{DirectCompute}

\section{OpenGL Compute Shader}

\section{OpenMP}
\chapter{Implementation}

The FFT application has been implemented in C/C++, CUDA, OpenCL, DirectCompute and OpenGL on a GeForce GTX 670 and Radeon R7 260X graphics card and a Core i7 3770K 3.5GHz CPU.

\section{Benchmark application GPU}

\subsection{FFT}

\subsubsection{Setup}

The implementation of the FFT algorithm on a GPU can be broken down into a few steps, see figure \ref{fig:algorithm-overview} for a simplified overview. The application setup differs among the tested technologies, however some steps can be generalized; get platform and device information, allocate device buffers and upload data to device.
\begin{figure}
	\centering
	\includestandalone[width=\textwidth]{figures/overview}
	\caption{Overview of the events in the algorithm.}
	\label{fig:algorithm-overview}
\end{figure}

The next step is to calculate the specific FFT arguments for a $N$-point sequence for each kernel. The most important difference between devices and platforms are local memory capacity and thread and block configuration. Threads per block was selected for the best performance. See table \ref{tab:threads-per-block} for details.
\begin{table}
	\centering
	\includestandalone[width=\textwidth]{tables/threadsperblock}
	\caption{Shared memory size in bytes, threads and block configuration per device.}
	\label{tab:threads-per-block}
\end{table}

\subsubsection{Butterfly}

The implementation of a $N$-point radix-2 FFT algorithm have $\log_2 N$ stages with $N/2$ butterfly operations per stage. A butterfly operation is an addition, a subtraction, followed by a multiplication by a twiddle factor, see figure \ref{fig:butterfly}.
\begin{figure}[h]
	\centering
	% FFT Butterfly
\tikzstyle{n}= [circle, fill, minimum size=4pt,inner sep=0pt, outer sep=0pt]
\tikzstyle{mul} = [circle,draw,inner sep=-1pt]

% Define two helper counters
\newcounter{x}\newcounter{y}
\begin{tikzpicture}[%
	yscale=0.5,
	xscale=1.2,
	node distance=0.3cm,
	auto]
    % The strategy is to create nodes with names: N-column-row
    % Input nodes are named N-0-0 ... N-0-15
    % Output nodes are named N-10-0 ... N-10-15

    % Draw inputs
    \foreach \y in {0,...,15}
        \node[n, pin={[pin edge={latex'-,black}]left:$x(\y)$}] (N-0-\y) at (0,-\y) {};
              
    % Draw outputs
    \foreach \y / \idx in {0,...,15}
        \node[n, pin={[pin edge={-latex',black}]right:$X(\y)$}] (N-11-\y) at (8,-\y) {};
              
   % draw connector nodes
    \foreach \y in {0,...,15}
        \foreach \x / \c in {1/1,2/3,3/4,4/6,5/7,6/9,7/10}
            \node[n, name=N-\c-\y] at (\x,-\y) {};
            
    % draw x nodes
    \foreach \y in {0,...,7}
        \foreach \x / \c  in {1/2}
            \node[mul, right of=N-\x-\y] (N-\c-\y) {};            
    \foreach \y in {8,...,15}
        \foreach \x / \c  in {1/2}
            \node[mul, right of=N-\x-\y] (N-\c-\y) {${\times}$};
    % 
    \foreach \y in {0,...,3}
        \foreach \x / \c  in {4/5}
            \node[mul, right of=N-\x-\y] (N-\c-\y) {};
    \foreach \y in {4,...,7}
        \foreach \x / \c  in {4/5}
            \node[mul, right of=N-\x-\y] (N-\c-\y) {${\times}$};
    \foreach \y in {8,...,11}
        \foreach \x / \c  in {4/5}
            \node[mul, right of=N-\x-\y] (N-\c-\y) {};
    \foreach \y in {12,...,15}
        \foreach \x / \c  in {4/5}
            \node[mul, right of=N-\x-\y] (N-\c-\y) {${\times}$};
    % 
    \foreach \y in {0,2,4,6,8,10,12,14}
        \foreach \x / \c  in {7/8}
            \node[mul, right of=N-\x-\y] (N-\c-\y) {};
    \foreach \y in {1,3,5,7,9,11,13,15}
        \foreach \x / \c  in {7/8}
            \node[mul, right of=N-\x-\y] (N-\c-\y) {${\times}$};    

    % horizontal connections
    % Note the use of simple counter arithmetics to get correct
    % indexes.
    \foreach \y in {0,...,15}
    {
		\foreach \x in {0,1,3,4,7}
		{
			\setcounter{x}{\x}\stepcounter{x}
			\path (N-\x-\y) edge[-] (N-\arabic{x}-\y);
		}
	}
       
    % Draw the W_16 coefficients
    \setcounter{y}{0}
    \foreach \i in {0,...,7}
    {
	   	\path (N-2-\arabic{y}) edge[-] node {} (N-3-\arabic{y});
	    \stepcounter{y}
    }
    \foreach \i in {0,...,7}
    {
    	\path (N-2-\arabic{y}) edge[-] node {\tiny $W^{\i}_{16}$} (N-3-\arabic{y});
        \stepcounter{y}
    }
    
    % Draw the W_8 coefficients
    \setcounter{y}{0}
    \foreach \tmp in {0,1}
	{
    	\foreach \i in {0,...,3}
    	{
        	\path (N-5-\arabic{y}) edge[-] node {} (N-6-\arabic{y});
        	\addtocounter{y}{1}
    	}
    	\foreach \i in {0,...,3}
    	{
        	\path (N-5-\arabic{y}) edge[-] node {\tiny $W^{\i}_{8}$} (N-6-\arabic{y});
        	\addtocounter{y}{1}
    	}
    }

    % Draw the W_4 coefficients
    \setcounter{y}{0}
	\foreach \tmp in {0,...,3}
	{    
		\foreach \i in {0,1}
		{
			\path (N-8-\arabic{y}) edge[-] node {} (N-9-\arabic{y});
			\stepcounter{y}
			\path (N-8-\arabic{y}) edge[-] node {\tiny $W^{\i}_{4}$} (N-9-\arabic{y});
			\stepcounter{y}
		}
    }
    
    % Connect nodes
    \foreach \sourcey / \desty in {	0/8,	1/9,	2/10,	3/11,
									4/12,	5/13,	6/14,	7/15,
									8/0,	9/1,	10/2,	11/3,
									12/4,	13/5,	14/6,	15/7}
       \path (N-0-\sourcey.east) edge[-] (N-1-\desty.west);
    \foreach \sourcey / \desty in {	0/4,	1/5,	2/6,	3/7,
									4/0,	5/1,	6/2,	7/3,
									8/12,	9/13,	10/14,	11/15,
									12/8,	13/9,	14/10,	15/11}
        \path (N-3-\sourcey.east) edge[-] (N-4-\desty.west);
    \foreach \sourcey / \desty in {	0/0,	1/2,	2/0,	3/2,
    								0/1,	1/3,	2/1,	3/3,
                                   	4/4,	5/6,	6/4,	7/6,
                                   	4/5,	5/7,	6/5,	7/7,
                                   	8/8,	9/10,	10/8,	11/10,
									8/9,	9/11,	10/9,	11/11,
									12/12,	13/14,	14/12,	15/14,
									12/13,	13/15,	14/13,	15/15}
	{
        \path (N-6-\sourcey.east) edge[-] (N-7-\desty.west);
        \path (N-9-\sourcey.east) edge[-] (N-10-\desty.west);
    }
    % Nodes are in bit-reverse order
    \foreach \sourcey / \desty in {	0/0,1/8,2/4,3/12,4/2,5/10,6,7/14,8/1,9,10/5,11/13,12/3,13/11,14/7,15/15}
	{
        \path (N-10-\sourcey.east) edge[-] (N-11-\desty.west);
    }
    
    % Add region boxes
	% Partial stage
	\def \lastNode {10}
	\node[draw,dashed,fit=(N-6-0) (N-\lastNode-3)] {};
	\node[draw,dashed,fit=(N-6-4) (N-\lastNode-7)] {};
	\node[draw,dashed,fit=(N-6-8) (N-\lastNode-11)] {};
	\node[draw,dashed,fit=(N-6-12) (N-\lastNode-15)] {};	
    % Complete stage
	\node[draw,densely dotted,fit=(N-0-0) (N-2-15),label=above:{stage 1}] {};
	\node[draw,densely dotted,fit=(N-3-0) (N-5-15),label=above:{stage 2}] {};
	\node[draw,fit=(N-6-0) (N-8-15),opacity=0,label=above:{stage 3},name=Stage-3] {};
	\node[draw,fit=(N-9-0) (N-\lastNode-15),opacity=0,label=above:{stage 4},name=Stage-4] {};
	\node[draw,fit=(N-11-0) (N-11-15),opacity=0,label=above:{output}] {};
	\node[draw,densely dotted,fit=(Stage-3) (Stage-4)] {};
	\node[draw,fit=(N-\lastNode-0) (N-11-15)] {};
\end{tikzpicture}
	\caption{Butterfly operations}
	\label{fig:butterfly}
\end{figure}

\subsubsection{Thread and block scheme}

The threading scheme was one butterfly per thread, so that a sequence of sixteen points require eight threads. Each platform was configured to a number of threads per block (see table \ref{tab:threads-per-block}) and any sequences requiring more butterfly operations then the threads per block configuration needed the computations to be split over several blocks. In the case of sequence exceeding one block, the sequence is mapped over the \texttt{blockIdx.y} dimension with size \texttt{gridDim.y}. The block dimension \texttt{blockIdx.x} is used to calculate sequence id when running a batch of sequences, the block dimensions are limited to $2^{31}$, $2^{16}$, $2^{16}$ respectively for \texttt{x}, \texttt{y}, \texttt{z}. Example: if the threads per block limit is two, then four blocks would be needed for a sixteen point sequence.
\begin{figure}
	\input{figures/exampleflow}
	\caption{Example flow graph of a sixteen-point FFT using (stage 1 and 2) Cooley-Tukey algorithm and (stage 3 and 4) constant geometry algorithm. The solid box is the bit-reverse order output. Dotted boxes are separate kernel launches, dashed boxes are data transfered to local memory before computing the remaining stages.}
	\label{fig:flowgraph-16}
\end{figure}

\subsubsection{Synchronization}

Thread synchronization is only available within a block. When the sequence or partial sequence fitted within a block all data was transferred to local memory before computing the last stages. If the sequence was larger and required more then one block the synchronization was handled by launching several kernels in the same stream to be executed in sequence. The kernel launched for block wide synchronization is called the global kernel and the kernel for thread synchronization within a block is called the local kernel. The global kernel had an implementation of the Cooley-Tukey FFT algorithm and the local kernel had constant geometry (same indexing for every stage). The last stage outputs from shared memory in bit reversed order to the global memory. See figure \ref{fig:flowgraph-16} where the sequence length is 16 and the threads per block is set to two.

\subsubsection{Calculation}

The indexing for the global kernel was calculated from the thread id and block id (\texttt{threadIdx.x} and \texttt{blockIdx.x} in CUDA) as seen in figure \ref{fig:code-global-index}. Input and output is done on the same index.
\begin{figure}
	\centering
	\lstset{language=C++}
	\begin{framed}
	\begin{lstlisting}
int tid     = blockIdx.x * blockDim.x + threadIdx.x,
    io_low  = tid + (tid & (0xFFFFFFFF << stages_left)),
    io_high = index1 + (N >> 1);
	\end{lstlisting}
	\end{framed}
	\caption{ CUDA example code showing index calculation for each stage in the global kernel, N is the total number of points. \texttt{io\_low} is the index of the first input in the butterfly operation and \texttt{io\_high} the index of the second.}
	\label{fig:code-global-index}
\end{figure}

Index calculation for the local kernel is done once for all stages, see figure \ref{fig:code-local-index}. These indexes are separate from the indexing in the global memory. The global memory offset depends on threads per block (\texttt{blockDim.x} in CUDA) and block id.
\begin{figure}
	\centering
	\lstset{language=C++}
	\begin{framed}
	\begin{lstlisting}
int n_per_block = N / gridDim.x.
    in_low      = threadId.x.
    in_high     = threadId.x + (n_per_block >> 1).
    out_low     = threadId.x << 1.
    out_high    = out1 + 1;
	\end{lstlisting}
	\end{framed}
	\caption{ CUDA example code showing index calculation for points in shared memory for the CUDA local kernel. }
	\label{fig:code-local-index}
\end{figure}

The last operation after the last stage is to perform the bit-reverse indexing operation, this is done when writing from shared to global memory. The implementation of bit reversing is available as a intrinsic integer instruction, see table \ref{tab:bit-reverse-intrinsics}. If instruction is not available figure \ref{fig:code-bit-reverse} shows the code used. The bit reversed value had to be right shifted the number of zeroes leading the number in a 32-bit int. Example of the bit-reverse index operation: index 8 of a 16 point sequence is bit-reversed to 1, in binary its 1000 reversed to 0001. Index 8 of a 32 point sequence is bit-reversed to 2, corresponds to 01000 to 00010. Figure \ref{fig:flowgraph-16} show the complete bit-reverse operations of a 16-point sequence in the output step after the last stage.

\begin{table}[h!]
	\centering
	\includestandalone[width=\textwidth]{tables/bit-reverse-intrinsics}
	\caption{Integer intrinsic bit-reverse function for different technologies.}
	\label{tab:bit-reverse-intrinsics}
\end{table}

\begin{figure}[h]
	\centering
	\lstset{language=C++}
	\begin{framed}
	\begin{lstlisting}
x = (((x & 0xaaaaaaaa) >> 1) | ((x & 0x55555555) << 1));
x = (((x & 0xcccccccc) >> 2) | ((x & 0x33333333) << 2));
x = (((x & 0xf0f0f0f0) >> 4) | ((x & 0x0f0f0f0f) << 4));
x = (((x & 0xff00ff00) >> 8) | ((x & 0x00ff00ff) << 8));
return((x >> 16) | (x << 16));
	\end{lstlisting}
	\end{framed}
	\caption{ Code returning a bit reversed unsigned integer where x is the input. Only 32-bit integer input and output. }
	\label{fig:code-bit-reverse}
\end{figure}

\subsection{FFT 2D}

The FFT algorithm for two dimensional data, such as images, is first transformed row wise (each row as a separate sequence) and then a transform of each column. The implementation performs a row wise transformation and then transposes the whole image twice, see figure \ref{lst:cuda:host-2d-example}. A transformed image is shown in figure \ref{fig:twodimentransform}.

\begin{figure}[!h]
	\centering
	\begin{framed}
		\includestandalone[width=\textwidth]{code/cuda-host-2d}	
	\end{framed}
	\caption{CUDA host code for the 2D FFT algorithm.}
	\label{lst:cuda:host-2d-example}	
\end{figure}

\begin{figure}
	\centering
	\subfloat[Original image\label{image-1:lena}]{
		\includegraphics[keepaspectratio=true, scale=0.33]{images/lena.jpg}{}		
    }
    \hfill
    \subfloat[Magnitude representation\label{image-2:lena}]{
		\includegraphics[keepaspectratio=true, scale=0.33]{images/lena_transformed.jpg}{}
    }
	\caption{Original image \ref{image-1:lena} transformed and represented with a quadrant shifted magnitude visualization \ref{image-2:lena}. }
    \label{fig:twodimentransform}
\end{figure}

The difference between the FFT kernel for 1D and 2D are the indexing scheme. 2D are indexed as rows at \texttt{blockIdx.x} and columns at $\texttt{threadIdx.x} + \texttt{blockIdx.y} \cdot \texttt{blockDim.x}$. For 2D \texttt{blockIdx.z} is used as the sequence id in a batch.

\subsubsection{Transpose}

The transpose kernel uses a different index mapping of the 2D-data and blocks/threads then the FFT kernel. The data is tiled in a grid pattern where each tile represents one block, indexed by \texttt{blockIdx.x} and \texttt{blockIdx.y}. The tile size is a multiple of 32 for both dimensions and limited to the size of the shared memory buffer, see table \ref{tab:threads-per-block} for specific size per technology. To avoid banking issues, the last dimension is increased with one but not used. However, resolving the banking issue have little effect on total running-time so when shared memory is limited to 32768, the extra column is not used. The tiles rows and columns are diveded over the \texttt{threadIdx.x} and \texttt{threadIdx.y} index respectively. See figure \ref{lst:cuda:device-transpose} for code example.

Shared memory example: The CUDA shared memory can allocate 49152 bytes and a single data point require $\texttt{sizeof(float)} \cdot 2 = 8$ bytes. That leaves room for a tile size of $64 \cdot (64 + 1) \cdot 8 = 33280$ bytes.

\begin{figure}[h!]
	\centering
	\begin{framed}
		\includestandalone[width=\textwidth]{code/cuda-device-transpose}	
	\end{framed}
	\caption{CUDA device code for the transpose kernel.}
	\label{lst:cuda:device-transpose}	
\end{figure}

The transpose kernel uses the shared memory and tiling of the image to avoid large strides through global memory. Each block represents a tile in the image. The first step is to write the complete tile to shared memory and synchronize the threads before writing to the output buffer. Both reading from the input memory and writing to the output memory is performed in close stride. Figure \ref{fig:transpose-memory} shows how the transpose is performed in memory.

\begin{figure}[h!]
	\centering
	\includestandalone[width=\textwidth]{figures/transpose-tile}
	\caption{Illustration of how shared memory is used in transposing an image. Input data is tiled and each tile is written to shared memory and transposed before written to the output memory. }
	\label{fig:transpose-memory}
\end{figure}

\subsection{Differences}

\subsubsection{Setup}

The major differences was mostly related to the setup phase. The CUDA implementation is the most straight forward, \texttt{cudaMalloc(...)} to allocate a buffer and \texttt{cudaMemcpy(...)} to populate it. With CUDA you can write the device code in the same file as the host code and even share functions, the OpenCL and OpenGL require a \texttt{char *} buffer and is most practical if written in seperate file and read as a file stream to a \texttt{char} buffer. DirectCompute Shaders are most easily compiled from file.

Figure \ref{fig:code:setup} demonstrates a simplified overview of how to setup the kernels in the different technologies.
\begin{figure}
	\centering	
	\subfloat[OpenCL setup\label{setup:ocl}]{		
		\includestandalone[scale=0.9]{code/setup/cu}		
	}	
	\hfill
	\subfloat[OpenCL setup\label{setup:ocl}]{		
		\includestandalone[scale=0.9]{code/setup/ocl}		
	}
	\newline
	\subfloat[DirectCompute setup\label{setup:dx}]{		
		\includestandalone[scale=0.8]{code/setup/dx}		
	}
	\hfill
	\subfloat[OpenGL setup\label{setup:gl}]{		
		\includestandalone[scale=0.8]{code/setup/gl}		
	}
	\caption{A simplified overview of the different functions required for a kernel launch and upload of initial data. }		
	\label{fig:code:setup}
\end{figure}

\subsubsection{Kernel execution}

Kernel execution in CUDA is very much like any C-like language and the only difference is that the kernel launch syntax, $<<<$ and $>>>$, setting number of blocks and threads.

The other technologies require some more setup, OpenCL and OpenGL set one parameter per line. OpenCL maps with index whereas OpenGL maps with a string to the parameter. DirectCompute is best suited to use a constant buffer for the parameters, the constant buffer is read only and accessible globally over the kernel. DirectCompute and OpenGL share a similar launch style where the compute shader is set as the current program and then a dispatch call is made with the group (block) configuration. See table \ref{tab:kernel-execution} for a list of how the kernels are launched.

\begin{table}[h!]
	\centering
	\includestandalone[width=\textwidth]{tables/kernel-execution}
	\caption{Table illustrating how to set parameters and launch a kernel.}
	\label{tab:kernel-execution}
\end{table}

\subsubsection{Kernel code}

This is the part where the code differ the least on but a few points. CUDA have the strongest support for a C/C++ -like language and only adds a function type specifier. The kernel program is accessible from the host via the \texttt{\_\_global\_\_} specifier. OpenCL share much of this but restricted to a C99-style in the current version (2.0). One difference is how one can reference global and local buffers, these must be specified with the specifier \texttt{\_\_global} or \texttt{\_\_local}.

DirectCompute and OpenGL Compute Shader uses HLSL (High-Level Shading Language) and GLSL (OpenGL Shading Language) respectively. These languages are similar and share the same restrictions compared to CUDA C/C++ and OpenCL C-code. Device functions can not use pointers or recursion. However these are of little importance for the performance since all code is in-lined in the compilation/building of the kernel program.

\subsubsection{Synchronization}

The synchronization of threads and blocks is slightly different, CUDA have the options to synchronize threads within a block and to synchronize host and device, the equivalent exists in all but DirectCompute (DirectX 11) where the device synchronization is not done trivially as with a blocking function call. See table \ref{tab:kernel-synchronization} for the code used for each technology.

\begin{table}[h!]
	\centering
	\includestandalone[width=\textwidth]{tables/kernel-synchronization}
	\caption{Synchronize functions regarding within blocks/groups and between host and device. CUDA, OpenCL and DirectCompute uses the same kernel stream to run them sequentially; OpenGL uses the command \texttt{glMemoryBarrier(GL\_SHADER\_STORAGE\_BARRIER\_BIT)} to ensure kernels are run in the same order as launched.}
	\label{tab:kernel-synchronization}
\end{table}

\section{Benchmark application CPU}

\subsection{FFT with OpenMP}

The OpenMP implementation benefits in performance from calculating the twiddle factors in advance. The calculated values are stored in a buffer accessible from all threads. The next step is to calculate each stage of the FFT algorithm. Last is the output index calculation where elements are reordered. See figure \ref{fig:omp:overview} for an overview.

\begin{figure}
	\centering
	\includestandalone[width=\textwidth]{figures/omp-overview}
	\caption{OpenMP implementation overview transforming sequence of size $N$.}
	\label{fig:omp:overview}
\end{figure}

\subsubsection{Twiddle factors}

The twiddle factor are stored for each butterfly operation. To save time, only the real part are calculated and the imaginary part is retrieved from the real parts due to the fact that $\sin(x) = \cos(\pi/2 + x)$ and $\sin(\pi/2 + x) = -\cos(x)$ to store. See figure \ref{tab:omp:twiddle-overview} for an example. The calculations will be split among the threads by static scheduling in two steps, first calculate the real values, secondly copy from real to imaginary.

\begin{table}[h!]
	\centering
	\begin{tabular}{|l|l|r|}
	\multicolumn{3}{c}{Twiddle factor table $W$} \\ \hline
	$i$ & $\Re(W)$ & $\Im(W)$ \\ \hline
	0 & $\cos(\alpha \cdot 0)$ & $\Re(W[4])$ \\
	1 & $\cos(\alpha \cdot 1)$ & $\Re(W[5])$ \\
	2 & $\cos(\alpha \cdot 2)$ & $\Re(W[6])$ \\
	3 & $\cos(\alpha \cdot 3)$ & $\Re(W[7])$ \\	
	4 & $\cos(\alpha \cdot 4)$ & $-\Re(W[0])$ \\
	5 & $\cos(\alpha \cdot 5)$ & $-\Re(W[1])$ \\
	6 & $\cos(\alpha \cdot 6)$ & $-\Re(W[2])$ \\
	7 & $\cos(\alpha \cdot 7)$ & $-\Re(W[3])$ \\ \hline
\end{tabular}
	\caption{Twiddle factors for a 16-point sequence where $\alpha = (2 \cdot \pi) / 16$. Each row $i$ corresponds to the $i$th butterfly operation.}
	\label{tab:omp:twiddle-overview}
\end{table}

\subsubsection{Butterfly}

The same butterfly operation uses the constant geometry index scheme. The indexes are not stored from one stage to the next but it makes the output come in continues order. The butterfly operations are split among the threads by static scheduling.

\subsubsection{Bit Reversed Order}

See figure \ref{fig:omp:bit-reverse-order} for code showing the bit reverse ordering operation in C/C++ code.

\begin{figure}[h!]
	\centering
	\begin{framed}
		\includestandalone[width=\textwidth]{code/omp-bit-reverse}	
	\end{framed}
	\caption{ C/C++ code performing the bit reverse ordering of a N-point sequence. }
	\label{fig:omp:bit-reverse-order}
\end{figure}

\subsection{FFT 2D with OpenMP}

The implementation of 2D FFT with OpenMP run the transformations row wise and transposes the image and repeat. The twiddle factors are calculated once and stays the same.

\section{Benchmark configurations}

\subsection{Limitations}

All implementations are limited to handle sequences of $2^n$ length or $2^m \times 2^m$ where $n$ and $m$ are integers. The GPUs have a maximum of 2GB global memory available and the upper limit of 2D transformations are $m = 8192$ since the implementation uses two buffers and require $8192 \cdot 8192 \cdot \texttt{sizeof(float2)} \cdot 2 = 1073741824$ bytes, however the Radeon R7 260X card does not handle that size well and is limited to $m = 4096$ (and OpenGL is even limited to $m = 2048$).

\subsection{Testing}

All tests executed on the GPU utilize some implementation of event time stamps. The time stamp event retrieve the actual start of the kernel if the current stream is busy, if the time is measured by the host, some technologies are not able to synchronize and the timings may be very unstable. The CPU implementations used Windows \textit{QueryPerformanceCounter} function, which is a high resolution (<1{\micro}s) time stamp.

All tests on the GTX 670 are run in batches of constant size 1073741824 bytes. A batch with 256-point sequences would run a total of 262144 sequences.

The tests on the R7 260X was not stable on the same sizes and some technologies would fail on large sequences.

\subsection{Reference libraries}

A couple of reference libraries was included to compare how well the FFT implementation performed. See table \ref{tab:external-implementations}.

\begin{table}
	\centering
	\begin{tabular}{|l|l|l|l|}
	\hline
	Platform & Model & Library name & Version \\ \hline
	NVIDIA GPU & GeForce GTX 670 & cuFFT & 7.5 \\
	AMD GPU & Radeon R7 260X & clFFT & 2.8.0 \\ \hline
	Intel CPU & Core i7 3770K 3.5GHz & FFTW & 3.3.4 \\ \hline
\end{tabular}
	\caption{Libraries included to compare with the implementation.}
	\label{tab:external-implementations}
\end{table}

\chapter{Results and Evaluation}

\todo{Write short explanation of this chapter.}

\section{Performance}

\todo{Present all performance comparisons.}

\newcommand{\plotwidth}{{\textwidth} / 2 + 120pt}

\begin{figure}
	\centering
	\includestandalone[width=\plotwidth]{plots/gtx-overview}
	\caption{Broad overview of the results of measuring the time of a single forward transform on the GeForce GTX 670.}
	\label{fig:gtx:overview}
\end{figure}

\begin{figure}
	\centering
	\includestandalone[width=\plotwidth]{plots/r260x-overview}
	\caption{Broad overview of the results of measuring the time of a single forward transform on the \AMDCARD. The time measured of the clFFT library is at host synchronization.}
	\label{fig:r260x:overview}
\end{figure}

\begin{figure}
	\centering
	\includestandalone[width=\plotwidth]{plots/gtx-implementation}
	\caption{Performance relative CUDA implementation on \NVCARD.}
	\label{fig:gtx:implementation}
\end{figure}

\begin{figure}
	\centering
	\includestandalone[width=\plotwidth]{plots/r260x-implementation}
	\caption{Performance relative OpenCL implementation on \AMDCARD.}
	\label{fig:r260x:implementation}
\end{figure}

\begin{figure}
	\centering
	\includestandalone[width=\plotwidth]{plots/gtx-cpu}
	\caption{Performance relative CUDA implementation on \NVCARD and \INTELCPU.}
	\label{fig:gtx:cpu}
\end{figure}

\begin{figure}
	\centering
	\includestandalone[width=\plotwidth]{plots/gpu-comparison}
	\caption{Performance of respective implementation on the \AMDCARD and the \NVCARD.}
	\label{fig:gpu-comparison}
\end{figure}

\section{Qualitative assessment}

\todo{Present some soft values.}

\subsection{Scalability}

\todo{Present how well problems scale.}

\subsection{Portability}

\todo{Present how well code is ported to other platforms.}

\subsection{Programmability}

\todo{Present how easy or hard an algorithm is to implement.}


\part*{Appendix}
\appendix
%\include{chapter/details}
%\include{chapter/rtthesis-doc}

\backmatter

\bibliography{IEEEfull,chapter/library}

\printindex

\end{document}
